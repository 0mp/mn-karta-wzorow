\mnsection{Aproksymacja funkcji}

% Problem zadania najlepszej aproksymacji.
\entry
\textbf{ENA}:
$f \in F$
(pnią unormowana).
Szukamy ENA
$\enain{v}{V \subset F}$ takiego, że
$\forallin{v}{V} \norm{f - \ena{v}} \leq \norm{f-v}$;

% Fakt o istnieniu ENA.
\entry
$F$ pnią lin.,
$V \subset F$ podpnią lin. skończonego wymiaru, to
$\forall{f \in F} \ \exists{\enain{v}{V}}$;

% Normy.
\entry
Normy:
\subentry
$L^1(a,b)$:
$\norm{f}_1 \coloneqq \int^b_a\abs{f(t)} dt$;
\subentry
$L^2(a,b)$:
$\norm{f}_2 \coloneqq (\int^b_a\abs{f(t)}^2 dt)^{1/2}$;
\subentry
$C^0(a,b)$:
$\norm{f}_\infty \coloneqq \maxin{t}{[a,b]} \abs{f(t)}$;
\subentry
$L^2_\rho(a,b)$:
$\norm{f}_{2, \rho} \coloneqq (\int^b_a\abs{f(t)}^2 \rho(t) \ dt)^{1/2}$,
gdzie
$\rho(t) > 0$ na $[a,b]$
(pewna skończona liczba punktów, gdzie się zeruje);

% TODO: Przykładowe własności powyższych norm.

\entry
$t_{\text{opt}}, t_v \in [a,b]$:
\subentry
$l_1$ (dyskretna)
$\sum^n_{i=0} \abs{f(t_i)}$;
\subentry
$l_2$ (\dittotikz)
$(\sum^n_{i=0} \abs{f(t_i)}^2)^{1/2}$;
\subentry
$l_{C_0}$ (\dittotikz)
$\max_{i = 0,\ldots,n} \abs{f(t_i)}$;
\subentry
$l_2$ (dyskretna ważona)
$(\sum^n_{i=0} \rho_i\abs{f(t_i)}^2)^{1/2}, \rho_i>0$;

% Aproksymacja w przestrzeni unitarnej.
\entry
Niech $H$ to przestrzeń liniowa z iloczynem skalarnym
$(x,y)$ dla $x, y \in H$
\subentry
$H \coloneqq \mathbb{R}^n$:
$(x,y) \coloneqq x^T y$;
\subentry
$H \coloneqq C^0[a;b]$:
$(f,g) \coloneqq \int^b_a f(t)g(t) \ dt$;
\subentry
Iloczyn skalarny indukuje normę w $H$:
$\norm{f} \coloneqq \sqrt{(f,f)}$;

% Twierdzenie o charakteryzacji ENA w przestrzeni unitarnej.
% TODO: Przykłady przestrzeni unitarnych (np. Hilberta).
\entry
\textbf{Tw. (o charakteryzacji ENA w pni unitarnej)}:
Niech $V \subset H$ będzie podpnią liniową w $H$:
$\enain{v}{V}$ jest ENA dla $f \in H$
$\leftrightarrow$
$(f - \ena{v}, v) = 0 \ \forallin{v}{V}$;

% Wniosek z twierdzenia o charakteryzacji ENA w przestrzeni unitarnej.
\entry
Jeśli $V$ jest skończonego wymiaru, to ENA istnieje i jest jedyny.
Gdy $\set{v_i}^n_{i=1}$ są bazą pni $V$, to
$\ena{v} = \sum^n_{i=1} \ena{\alpha}_i v_i$
i współczynniki $\set{\ena{\alpha}_i}$
spełniają układ równań normalnych:
$\sum \ena{\alpha}_i (v_i, v_j) = (f, v_j)$;

\entry
$[(v_i, v_j)]^n_{i,j=1}$
jest symetryczna i dodatnio określona (macierz Grama),
więc ma dokładnie 1 rozwiązanie;

% Macierz Grama.
\entry
\textbf{M. Grama}: zazwyczaj gęsta i źle uwar., SPD (więc alg. Chol.);

% Macierz Grama dla przestrzeni z bazą ortogonalną.
\entry
Wybierając bazę ortogonalną w $V = \text{span}\set{p_i}^n_{i=1}$,
gdzie $(p_i, p_j) = 0, i \neq j$, dostajemy macierz Grama
$[(p_i, p_i)]$,
$\ena{v} = \sum((f,p_i)\cdot p_i)/(p_i, p_i)$,
a gdy $\set{p_i}^n_{i=1}$ unormowane:
$\ena{v} = \sum(f,p_i)\cdot p_i$;

% Definicja wielomianów ortogonalnych dla L^2_\rho.
\entry
$L^2_\rho(a;b)$:
$(f,g) \coloneqq \int^b_a f(x)g(x)\rho(x) \ dx$,
$\rho$ to funkcja wagowa $\rho > 0$ na $[a;b]$
(w skończonej liczbie punktów może się zerować);
$\norm{f}^2_\rho \coloneqq (f,f)^{1/2}$;

% Formuła trójczłonowa.
\entry
Wielomiany $\mathbb{P}_k, k=0,\ldots$ postaci
$P_k = x^k + o(x^k)$ wyrażają się wzorem:
$P_k(x) = (x + \beta_k) \cdot P_{k-1}(x) + \gamma_k P_{k-2}(x), k=1,\ldots$,
$P_{-1}(x)=0$,
$P_{0}(x)=1$,
$\beta_k= \frac{(x P_{k-1}, P_{k-1})}{\norm{P_{k-1}}^2}$,
$\gamma_k = \frac{(x P_{k-1}, P_{k-2})}{\norm{P_{k-2}}^2}$,
% TODO: Fragmenty dowodu.

% Wielomiany Legendre'a
\entry
Wielomiany Legendre'a:
ort. w $L^2(-1,1)$ z wagą $\rho = 1$:
$L_k(x) = \frac{2k-1}{k}xL_{k-1} - \frac{k-1}{k}L_{k-2}$,
$\norm{L_k}^2=\frac{2}{2k+1}$;

% Wielomiany Hermite'a.
\entry
Wielomiany Hermite'a:
$L^2_\rho(-\infty, \infty), \rho(x)=e^{-x^2}$,
$H_k(x) = 2xH_{k-1}(x) - (2k-2) H_{k-2}(x), H_0(x) \equiv 1$,
$\norm{H_k}^2 = \sqrt{\pi}2^kk!$;

% Wielomiany Czebyszewa.
\entry
Wielomiany Czebyszewa:
$L_\rho^2(-1,1), \rho(x)=\frac{1}{\sqrt{1-x^2}}$,
$T_k(x) = 2xT_{k-1}(x) - T_{k-2}(x), T_0(x)\equiv 1$,
$\norm{T_k}^2 = \pi \text{ if } k=0 \text{ else } \pi/2$;
