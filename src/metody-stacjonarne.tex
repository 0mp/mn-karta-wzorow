% Metody iteracyjne / stacjonarne.
\section{Metody stacjonarne}

% Ogólna idea metod stacjonarnych (1).
\entry
$A=M-Z$;
\entry
$Mx_{k+1}=b+Zx_{k}$
\entry
$x_{k+1}=x_k+M^{-1}r_k$, gdzie $r_k=b-Ax_k$;

% Ogólna idea metod stacjonarnych (2).
\entry
Metoda: $\mathrm{(**)}$ $x_{k+1}=M^{-1}(b+Zx_k)$,
aby $Mx_{k+1}=\tilde{b}$ było łatwe;

% Twierdzenie o zbieżności metody iteracyjnej (1).
\entry
Tw.: $\norm{M^{-1}Z} < 1$,
to metoda $\mathrm{(**)}$ zbieżna do rozw. $Ax=b \forall x_0 \in \mathbb{R}^N$

% Twierdzenie o zbieżności metody iteracyjnej (2).
\entry
Tw.: $\forall x_0$,
metoda $\mathrm{(**)}$
zbieżna do $x^* \leftrightarrow \rho(M^{-1}Z) < 1$,
gdzie $\rho(B) \coloneqq \max\set{\abs{\lambda} : \lambda\text{ --- w. wł. } B}$;

% Przykładowy algorytm implementujący iterację.
\entry
$\mathrm{iter}(A,b,x_0,M,t)$:
$x=x_0;
r=b - Ax;
(L,U,P) = lu(M);
\mathrm{while}(\norm{r} > t) \{
    p = M^{-1}r;
    x=x+p;
    r=b-Ax
\};
\mathrm{return}(x)
$;
$O(N^3 + \mathrm{\#iter} \cdot N^2)$;

% Metoda Jacobiego.
\entry
Metoda Jacobiego:
$M=\mathrm{diag}(A) = D$, $A=L+D+U$;
zbieżna, gdy $A$ ma dominująca przekątną,
tzn. $\abs{a_{ii}} > \sum_{j\neq i}\abs{a_{ij}} \forall i=1, \ldots, N$;

% Metoda Gaussa-Seidla.
\entry
Metoda Gaussa-Seidla:
$M = L + D$;
zbieżna, gdy $a_{ii}x_i^{k+1}=b_i-\sum_{j>i}a_{ij}x^k-\sum_{j<i}a_{ij}x^{k+1}$;

% Metoda Richardsona.
\entry
Metoda Richardsona:
$x_k+1 = x_k + \tau(b-Ax_k)$, gdzie $\tau\in\mathbb{R}$.
Gdy $A=A^T>0$,
to $\tau_{\mathrm{opt}}=\frac{2}{\lambda_{\mathrm{min}} + \lambda_{\mathrm{max}} }$
i $\gamma \coloneqq \norm{M^{-1}Z}=\abs{1-\lambda_{\mathrm{max}}\tau_{\mathrm{opt}}}=\frac{\mathrm{cond}_2(A)-1}{\mathrm{cond}_2(A) + 1}$;

% Algorytm iteracyjnego poprawiania w modelowej sytuacji.
\entry
Modelowe IR:
$\mathrm{for}(n=0,\ldots)\{ r_n=b-Ax_n; Ae_n=_{\mathrm{fl}}r_n; x_{n+1} = x_n + e_n \}$;

% TODO: Dodać dowody zbieżności modelowego algorytmu.

% Zbieżność modelowego algorytmu iteracyjnego poprawiania rozwiązania (IR, czyli iterative reconstruction)
\entry
Gdy $\mathrm{cond}(A)$ jest niezbyt duże względem $\frac{1}{\nu}$, to modelowy IR jest zbieżny;

