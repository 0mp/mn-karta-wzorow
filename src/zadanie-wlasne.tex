\section{Zadanie własne}

% TODO: Metody numeryczne rozwiązywania zagadnienia własnego.

% TODO: Metoda potęgowa.

\entry
Zadanie własne $A \in \mathbb{R}^{N \times N}$:
$(\lambda,x) \in \mathbb{C} \times \mathbb{C}^N: Ax = \lambda x$;

\entry
Gdy $\lambda$ dla $A$, to $(\lambda - \mu)$ dla $A-\mu I$;

\entry
Gdy $\lambda$ dla $A$ nieos., to $\frac{1}{\lambda}$ dla $A^{-1}$;

% TODO: Wniosek z powyższych faktów dla metody potęgowej.

% ^^^^^^^^^^^^^^^^^^^^^^^^^^^^^
% Zakres materiału do kolokwium.

% TODO: Odwrotna metoda potęgowa.

% TODO: Metoda Rayleigh (RQI).

% TODO: Fakty o lokalizacji wartości własnych.

% TODO: Twierdzenie Gershgorina.

% TODO: Twierdzenie (Bauer-Fike).

% TODO: Metoda QR wyznaczania wszystkich wartości własnych macierzy (wraz z ulepszeniem i wariantem praktycznym).
