\section{Zadanie własne}

% TODO: Metoda potęgowa.

% Definicja zadania własnego.
\entry
Zadanie własne $A \in \mathbb{R}^{N \times N}$:
$(\lambda,x) \in \mathbb{C} \times \mathbb{C}^N: Ax = \lambda x$;

% Fakt o wartościach własnych (1).
\entry
Gdy $\lambda$ dla $A$, to $(\lambda - \mu)$ dla $A-\mu I$;

% Fakt o wartościach własnych (2).
\entry
Gdy $\lambda$ dla $A$ nieos., to $\frac{1}{\lambda}$ dla $A^{-1}$;

% Wniosek z powyższych faktów dla metody potęgowej.
\entry
Met. potęgowa zastosowana do
$(A-\mu I)^{-1}$
będzie zbieżna do
$\lambda_i$,
czyli w. wł. najbliższej $\mu$.
Rzeczywiście, w. wł.
$(A-\mu I)^{-1}$,
to
$\frac{1}{\lambda_j-\mu}$
i jeśli
$\frac{1}{\abs{\lambda_i-\mu}} > \frac{1}{\abs{\lambda_j-\mu}}, j\neq i$,
to
$\frac{1}{\lambda_i-\mu}$
jest dominująca;

% ^^^^^^^^^^^^^^^^^^^^^^^^^^^^^
% Zakres materiału do kolokwium.

% TODO: Odwrotna metoda potęgowa.

% TODO: Metoda Rayleigh (RQI).

% TODO: Fakty o lokalizacji wartości własnych.

% TODO: Twierdzenie Gershgorina.

% TODO: Twierdzenie (Bauer-Fike).

% TODO: Metoda QR wyznaczania wszystkich wartości własnych macierzy (wraz z ulepszeniem i wariantem praktycznym).
