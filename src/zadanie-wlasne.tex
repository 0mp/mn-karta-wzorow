\mnsection{Zadanie własne}

% Metoda potęgowa wyznaczania wektora własnego.
% (9 XI/4.5/43)
\entry
M. pot.:
$\abs{\lambda_1}>\abs{\lambda_2}\geq\ldots\geq\abs{\lambda_n}$
$\Rightarrow$
$\mathrm{for}(k=0,\dots)\{x_{k+1}=Ax_k / \norm{x_{k+1}}^2_2\}$;

% Definicja zadania własnego.
\entry
Zadanie własne $A \in \mathbb{R}^{N \times N}$:
$(\lambda,x) \in \mathbb{C} \times \mathbb{C}^N$:
$Ax = \lambda x$;

% Fakt o wartościach własnych (1).
\entry
Gdy $\lambda$ dla $A$, to $(\lambda - \mu)$ dla $A-\mu I$;

% Fakt o wartościach własnych (2).
\entry
Gdy $\lambda$ dla $A$ nieos., to $\frac{1}{\lambda}$ dla $A^{-1}$;

% Wniosek z powyższych faktów dla metody potęgowej.
\entry
Met. potęgowa zastosowana do
$(A-\mu I)^{-1}$
będzie zbieżna do
$\lambda_i$,
czyli w. wł. najbliższej $\mu$.
Rzeczywiście, w. wł.
$(A-\mu I)^{-1}$,
to
$\frac{1}{\lambda_j-\mu}$
i jeśli
$\frac{1}{\abs{\lambda_i-\mu}} > \frac{1}{\abs{\lambda_j-\mu}}, j\neq i$,
to
$\frac{1}{\lambda_i-\mu}$
jest dominująca;

% ^^^^^^^^^^^^^^^^^^^^^^^^^^^^^
% Zakres materiału do kolokwium.

% Odwrotna metoda potęgowa.
% (MP: 16 XI/1.5/2)
\entry
\textbf{Odw. m. pot.}:
$\text{for } k=0,\ldots$:
$x_{k+1} = (A-\mu I)^{-1}x_k$;
$x_{k+1} = x_{k+1} / \norm{x_{k+1}}$;
$\mathcal{O}(n^3)$;
% Praktyczny algorytm implementujący odwrotna metodę potęgową.
\entry
Szybciej:
$P(A- \mu I = LU$ (lub $A-\mu I = QR$) ($\mathcal{O}{n^3}$);
$\text{for } k=0,\ldots$:
$Ly=Px_k$;
$Ux_{k+1} = y$;
$x_{k+1} = x_{k+1} / \norm{x_{k+1}}$;
($\mathcal{O}(kN^2)$)

% Najlepsze przybliżenie wartości własnej z użyciem ilorazu Rayleigh.
% (MP: 16 XI/1.5/23)
\entry
Znając przybliżony $x_k$ wek. wł.: mamy najlepsze (w sensie
średniokwadratowym) przybliżenie war. wł. (\textbf{iloraz Rayleigh}):
$\lambda_k \coloneqq x_k^T A x_k / x_k^T x_k$.
Gdy $x_k = v$:
$v^T A v / v^T v = v^T \lambda v / v^T v = \lambda$;

% Metoda Rayleigh (RQI).
% (MP: 16 XI/2/24)
\entry
\textbf{M. Rayleigh} (RQI):
$\text{for } k=0,\ldots$:
$x_{k+1} = (A - \mu_k I)^{-1} x_k$;
$x_{k+1} = x_{k+1} / \norm{x_{k+1}}$;
$\mu_{k+1} = x_{k+1}^T A x_{k+1}$.
Zbiega $\cdot^3$, a coraz gorsze uwarunkowanie ($A-\mu_k I$) pomaga.
Po $3$ iteracjach precyzja arytmetyki.

% TODO: Wartości własne macierzy blokowo-trójkątnej.
% (MP: 16 XI/2.5/3)

% Fakty o lokalizacji wartości własnych.
% (MP: 16 XI/3/4)
\entry
F. o lok. war. wł.:
$\abs{\lambda} \leq \norm{A}$;
% Definicja \sigma(A).
\entry
Zb. wartości wł. $A$:
$\sigma(A)$;

% Twierdzenie Gerszgorina.
\entry
Tw. Gerszgorina:
$\sigma(A) \in  \sum$ kół:
$K_i \coloneqq \set{z\in\mathbb{C}: \shortabs{z - a_{ii}} \leq \sum_{j \neq i}\shortabs{a_{ij}}}$;

% TODO: Twierdzenie (Bauer-Fike).

% TODO: Metoda QR wyznaczania wszystkich wartości własnych macierzy (wraz z ulepszeniem i wariantem praktycznym).
