\mnsection{Równania nieliniowe}

% TODO: Twierdzenie Darboux.

% TODO: Twierdzenie o przedziałach lokalizujących zero w kolejnych krokach metody bisekcji.

\entry
Szybkość zbieżności:
wykładnicza$(p)$:
$\exists_{c \geq 0} \exists_{N > 0}
\abs{x_{n + 1} - x^{*}} = c \abs{x_n - x^{*}}^p \forall_{n \geq N}$;
liniowa$(\gamma)$:
$\exists_{\gamma \in [0;1)} \exists_{N > 0}
\abs{x_{n + 1} - x^{*}} = \gamma \abs{x_n - x^{*}} \forall_{n \geq N}$;

% TODO: Metoda Newtona: czy tyle wystarczy?
\entry
Metoda Newtona:
$f(x) = 0 = f(x+h) = f(x) + f'(x)h + (f''(x)h^2/2+\cdots$;
$x_{n+1} = x_n - f(x_n)/f'(x_n)$;

% Twierdzenie o zbieżności metody Newtona.
\entry
Tw. (zbieżność m. N.):
$f \in C^2[a;b], \exists_{\ena{x} \in (a;b)}$,
że $f(\ena{x})=0, f'(\ena{x}) \neq 0$.
Wtedy $\exists_{U \ni \ena{x}}$, że m. N. zbieżna do $\ena{x} \ \forall_{x_0 \in U}$.
Ponadto zbieżność kwadratowa:
$\exists_{c>0} \abs{x_{n+1} - \ena{x}} \leq c \abs{x_n - \ena{x}}^2 \forall_{n=0,\ldots}$;

% Własności metody Newtona.
\entry
Własności m. N.:
\subentry
$f$ musi być różniczkowalna;
\subentry
musi być znany wzór na pochodną;
\subentry
$f'(\ena{x}) \neq 0 \Rightarrow \ena{x}$ zerem jednokrotnym i gdy krotności $m$, 
to metoda zbiega liniowo do 
$\abs{x_{n+1} - \ena{x}} \approx (1-1/m) \cdot \abs{x_n - \ena{x}}$;

% Metoda Newtona dla 1/a.
\entry
M. N. dla
$\frac{1}{a}$: $x_{n+1} = x_n(2 - x_na)$, $0<x_0<\frac{2}{a}$;

% TODO: Metoda Herona.

% TODO: Metoda siecznych.

% Metoda Steffensena.
\entry
M. Steffensena:
$x_{n+1} = x_n - f(x_n) \cdot \frac{f(x_n)}{f(x_n + f(x_n)) - f(x_n)}, p=2$,
(problemy z niedomiarem);

% TODO: Metoda cięciw.

% TODO: Metoda Halley'a.

% TODO: Metoda typu punktu stałego.

% Twierdzenie Banacha o kontrakcji
\entry
\textbf{Tw. Banacha o kontrakcji}:
$F:[a;b] \Rightarrow [a;b]$
i spełnia warunki kontrakcji:
$\exists_{\gamma < 1} \forallin{x,y}{[a;b]} F(\ena{x}) = \ena{x}$
oraz ciąg 
$x_{n+1} = F(x_n)$ 
jest zbieżny liniowo 
$\forallin{x_o}{[a;b]}$
do $\ena{x}$:
$\abs{x_{n+1} - \ena{x}} \leq \gamma \abs{x_n - \ena{x}}$;
