\section{Równania nieliniowe}

% TODO: Twierdzenie Darboux.

% TODO: Twierdzenie o przedziałach lokalizujących zero w kolejnych krokach metody bisekcji.

\entry
Szybkość zbieżności:
wykładnicza$(p)$:
$\exists_{c \geq 0} \exists_{N > 0}
\abs{x_{n + 1} - x^{*}} = c \abs{x_n - x^{*}}^p \forall_{n \geq N}$;
liniowa$(\gamma)$:
$\exists_{\gamma \in [0;1)} \exists_{N > 0}
\abs{x_{n + 1} - x^{*}} = \gamma \abs{x_n - x^{*}} \forall_{n \geq N}$;

% TODO: Metoda Newtona: czy tyle wystarczy?
\entry
Metoda Newtona:
$f(x) = 0 = f(x+h) = f(x) + f'(x)h + (f''(x)h^2/2+\cdots$;
$x_{n+1} = x_n - f(x_n)/f'(x_n)$;

% Twierdzenie o zbieżności metody Newtona.
\entry
Tw. (zbieżność m. N.):
$f \in C^2[a;b], \exists \ena{x} \in (a;b)$,
że $f(\ena{x})=0, f'(\ena{x}) \neq 0$.
Wtedy $\exists_{U \ni \ena{x}}$, że m. N. zbieżna do $\ena{x} \ \forall_{x_0 \in U}$.
Ponadto zbieżność kwadratowa:
$\exists_{c>0} \abs{x_{n+1} - \ena{x}} \leq c \abs{x_n - \ena{x}}^2 \forall_{n=0,\ldots}$;

% TODO: Metoda Herona.

% TODO: Metoda siecznych.

% TODO: Metoda Seffensena.

% TODO: Metoda cięciw.

% TODO: Metoda Halley'a.

% TODO: Metoda typu punktu stałego.

% TODO: Twierdzenie Banacha o kontrakcji
