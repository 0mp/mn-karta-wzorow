\section{Równania nieliniowe}

% TODO: Twierdzenie Darboux.

% TODO: Twierdzenie o przedziałach lokalizujących zero w kolejnych krokach metody bisekcji.

\entry
Szybkość zbieżności:
wykładnicza$(p)$:
$\exists_{c \geq 0} \exists_{N > 0}
\abs{x_{n + 1} - x^{*}} = c \abs{x_n - x^{*}}^p \forall_{n \geq N}$;
liniowa$(\gamma)$:
$\exists_{\gamma \in [0;1)} \exists_{N > 0}
\abs{x_{n + 1} - x^{*}} = \gamma \abs{x_n - x^{*}} \forall_{n \geq N}$;

% TODO: Metoda Newtona: czy tyle wystarczy?
\entry
Metoda Newtona:
$f(x) = 0 = f(x+h) = f(x) + f'(x)h + (f''(x)h^2/2+\cdots$;
$x_{n+1} = x_n - f(x_n)/f'(x_n)$;

% TODO: Twierdzenie o zbieżności metody Newtona.

% TODO: Metoda Herona.

% TODO: Metoda siecznych.

% TODO: Metoda Seffensena.

% TODO: Metoda cięciw.

% TODO: Metoda Halley'a.

% TODO: Metoda typu punktu stałego.

% TODO: Twierdzenie Banacha o kontrakcji
