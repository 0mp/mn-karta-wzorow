\section{Uwarunkowanie zadania}

\entry
Wskaźnik u. $P$ w punkcie $x$:
$\mathrm{cond}_{\mathrm{abs}}(P,x) \coloneqq \sup\limits_{\text{małe }\delta} \frac{\norm{P(x+\delta) - P(x)}}{\norm{\delta}}$;

\entry
$\norm{P(x+\delta)} \leq \mathrm{cond}_{\mathrm{abs}}(P,x)\norm{\delta}$;

\entry
$\frac{\norm{P(x+\delta) - P(x)}}{\norm{P(x)}} \leq \mathrm{cond}_{\mathrm{rel}}(P,x)\frac{\norm{\delta}}{\norm{x}}$;

% Idealizacja.

\entry
$\mathrm{cond}_{\mathrm{abs}}(P,x) \coloneqq \lim\limits_{\norm{\delta \to 0}} \frac{\norm{P(x+\delta) - P(x)}}{\norm{\delta}}$;

\entry
$\mathrm{cond}_{\mathrm{rel}}(P,x) \coloneqq \mathrm{cond}_{\mathrm{abs}}(P,x)\frac{\norm{x}}{\norm{P(x)}} $;

\entry
Zadanie $P$ jest źle uwarunkowane w punkcie $x$,
gdy $\mathrm{cond}(P,x) \gg 1$,
bo małe zaburzenie danych może spowodować duży błąd wyniku;

% XXX: Czy to powinny być normy drugie?
\entry
$\mathrm{cond}(A) \coloneqq \norm{\vphantom{A^{-1}}A}\cdot\norm{A^{-1}}$;

\entry
$A=A^T\in\mathbb{R}^{N\times N}$,
to $\mathrm{cond}_2(A)\coloneqq \frac{\max_i\abs{\lambda_i}}{\min_i\abs{\lambda_i}}$;

\entry
$\exists_{v_i\neq 0} A_{v_i} = \lambda_i v_i$,
gdzie $v_i$ --- wektor własny;

\entry
Gdy
$\epsilon\cdot\mathrm{cond}(A) < \frac{1}{2}$,
to $\frac{\norm{x - \tilde{x}}}{\norm{x}} \leq 4 \cdot \mathrm{cond}(A) \cdot \epsilon$;

\entry
Jeśli $\norm{B} < 1$,
to $I+B$ odwracalna i $\norm{(I+B)^{-1}} \leq 1 / (1 - \norm{B})$;

\entry
Jeśli $A$ odwracalna i $\norm{A^{-1}\Delta} < 1$,
to $(A+\Delta)^{-1}$ istnieje
i $\norm{(A+\Delta)^{-1}} \leq \frac{\norm{A^{-1}}}{1 - \norm{A^{-1}} \cdot \norm{\vphantom{A^1}\Delta}}$,
gdzie $\Delta \leq \epsilon\norm{A}$;

% TODO: Fakt o rozwiązaniu \tilde{x} układu Ax=b za pomocą LU z wyborem w kolumnie.

% Numeryczne kryterium NP.
\entry
Numeryczne kryterium NP:
gdy $\tilde{x}$ przybliżonym rozw. $Ax=b$,\\
to $(A+\Delta)\tilde{x}=b+\delta$
i $\frac{\norm{\delta}}{\norm{b}}, \frac{\norm{\Delta}}{\norm{A}} \leq \epsilon$,
gdzie $\epsilon \coloneqq \frac{\norm{b-A\tilde{x}}}{\norm{A} \cdot \norm{\tilde{x}} + \norm{b} }$;

% TODO: Przykłady zadań dobrze i źle uwarunkowanych.

% Lemat.
\entry
$\norm{\Delta} < 1$,
to $I + \Delta$ nieosobliwa
oraz $\norm{(I + \Delta)^{-1}} \leq \frac{1}{1 - \norm{\Delta}}$;
