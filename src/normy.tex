\mnsection{Normy}

% Normy wektorowe w $\mathbb{R}^n$.

\entry
$\norm{x}_1 \coloneqq \sum^n_{i=1}\abs{x_i}$;
\entry
$\norm{x}_2 \coloneqq \sqrt{x_1^2+\ldots + x_n^2}$;
\entry
$\norm{x}_\infty \coloneqq \max_{i}\abs{x_i}$;

\entry
$\norm{x}_p \coloneqq (\sum_i\abs{x_i}^p)^{1/p}$;
% Nierówność Cauchy’ego-Schwarza.
\entry
$\abs{x^Ty}_2 \leq \norm{x}_2\cdot\norm{y}_2$, dla $x,y\in\mathbb{R^N}$;

\entry
$A^TA=I$, to $\norm{Ax}^2_2=\norm{x}^2_2$;
\entry
$\norm{\mathrm{diag}} = \max_i d_i$;
%\entry
%$\norm{I} = 1$;

% Normy macierzowe indukowane przez normy wektorowe.

% Fakty.
%\entry
%$\norm{Ax} \leq \norm{A} \cdot \abs{x}$;
%\entry
%$\norm{AB} \leq \norm{A} \cdot \norm{B}$;

\entry
$\norm{A}_1 \coloneqq \max_j\sum_i\abs{a_{ij}}$;
\entry
$\norm{A}_2 \coloneqq \max\set{\sqrt{\lambda} : \lambda \in \sigma{A^TA}}$;

\entry
$\norm{A}_\infty \coloneqq \max_i\sum_j\abs{a_{ij}}$;
\entry
$\norm{A}_p \coloneqq \max\limits_{x\neq 0} \frac{\norm{A_x}_p}{\norm{x}_p} = \max\limits_{\norm{x}_p = 1} \norm{A_x}_p$;

\entry
Jeśli $\forall_{x} \norm{Ax}\leq c\norm{x}$, że $\exists_y \norm{Ay} = c\norm{y}$, to $c=\norm{A}$;

% Fakt z zadania 2.4. z części 2 ćwiczeń Konrada Sakowskiego.
\entry
$\norm{A}_2 = \sup_{\norm{x}_2=1, \norm{y}_2=1}\abs{y^TAx}$;
% Fakt z zadania 2.5. z części 2 ćwiczeń Konrada Sakowskiego.
\entry
$A\in\mathbb{R}^{N\times N} \Rightarrow \norm{A}^2_2=\norm{A^TA}_2$;

% Normy macierzowe nieindukowane przez normy wektorowe.

%\entry
%Norma Frobeniusa: $\norm{A}_F \coloneqq \sqrt{\sum_i\sum_j\abs{a_{ij}}^2}$; $\norm{I}_F=\sqrt{N}$;
