\section{Macierze rzadkie}

% Metody bezpośrednie.
\entry
Metody bezpośrednie:
r. LU z permutacją wierszy i kolumn, aby w czynnikach LU było dużo zer;

% TODO: Przykład strzałki Wilkinsona.

% Metody iteracyjne / stacjonarne.
\subsection{Metody stacjonarne}

% Ogólna idea metod stacjonarnych (1).
\entry
$A=M-Z$;
\entry
$Mx=b+Zx$;
\entry
$Mx_{k+1}=b+Zx_{k}$ (iteracja prosta);
\entry
$x_{k+1}=x_k+M^{-1}r-k$, gdzie $r_k=b-Ax_k$;

% Ogólna idea metod stacjonarnych (2).
\entry
Metoda: $\mathrm{(**)}$ $x_{k+1}=M^{-1}(b+Zx_k)$,
aby $Mx_{k+1}=\tilde{b}$ było łatwe;

% Twierdzenie o zbieżności metody iteracyjnej (1).
\entry
Tw.: $\norm{M^{-1}Z} < 1$,
to metoda $\mathrm{(**)}$ zbieżna do rozw. $Ax=b \forall x_0 \in \mathbb{R}^N$

% Twierdzenie o zbieżności metody iteracyjnej (2).
\entry
Tw.: $\forall x_0$,
metoda $\mathrm{(**)}$
zbieżna do $x^* \leftrightarrow \rho(M^{-1}Z) < 1$,
gdzie $\rho(B) \coloneqq \max\set{\abs{\lambda} : \lambda\text{ --- w. wł. } B}$;

% Metoda Jacobiego.
\entry
Metoda Jacobiego:
$M=\mathrm{diag}(A) = D$, $A=L+D+U$;
zbieżna, gdy $A$ ma dominująca przekątną,
tzn. $\abs{a_{ii}} > \sum_{j\neq i}\abs{a_{ij}} \forall i=1, \ldots, N$;

% Metoda Gaussa-Seidla.
\entry
Metoda Gaussa-Seidla:
$M = L + D$;
zbieżna, gdy $a_{ii}x_i^{k+1}=b_i-\sum_{j>i}a_{ij}x^k-\sum_{j<i}a_{ij}x^{k+1}$;

% Metoda Richardsona.
\entry
Metoda Richardsona:
$x_k+1 = x_k + \tau(b-Ax_k)$, gdzie $\tau\in\mathbb{R}$.
Gdy $A=A^T>0$,
to $\tau_{\mathrm{opt}}=\frac{2}{\lambda_{\mathrm{min}} + \lambda_{\mathrm{max}} }$;

\subsection{Metody przestrzeni Kryłowa}

% Definicja iteracji w metodach przestrzeni Kryłowa.
\entry
$k$-ta iteracja:
$x_k\in x_o+K_k$,
gdzie $K_k\coloneqq \set{r_0, Ar_0,\ldots, A^{k-1}r_0}$,
gdzie $r_0=b-Ax_0$;

% TODO: Coś tu jest nie tak.
% Metody oparte na minimalizacji błędu.
\entry
Metody oparte na minimalizacji błędu:
$x_k\in x_0 +K_k$ oraz $\norm{x_k - x^*}_B \forall x\in x_0 + K_k$;

% Metody oparte na minimalizacji residuum.
\entry
Metody oparte na minimalizacji residuum:
$\norm{b-Ax_k}_B \leq \norm{b-Ax}_B \forall x\in x_0 +K_k$, gdzie $B=B^T>0$;

% Metoda gradientów sprzężonych (CG).
\entry
Metoda gradientów sprzężonych (CG):
$A=A^T>0$, $x_k\in x_0 + K_k$ t.,
że $\norm{x_k+x^*}_A\leq \norm{x-x^*}_A \forall x\in x_0 + K_k$,
gdzie $\norm{y}^2_A \coloneqq y^TAy$.
Koszt iteracji, to mnożenie wektora przez $A$, czyli $O(N)$.
W idealnej arytmetyce zbieżne do $x^*$ w $\leq N$ iteracjach.
Po $k$ iteracjach:
$\norm{x_k - x^*}_A\leq 2(\frac{\sqrt{H} - 1}{\sqrt{H} + 1})^k\norm{x_0-x^*}_A$,
gdzie $H=\mathrm{cond}_2(A)$;

% TODO: Wzmianka o GMRES.
