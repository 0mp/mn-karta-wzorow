\section{Macierze rzadkie}

\entry
Metody bezpośrednie: r. LU z permutacją wierszy i kolumn, aby w czynnikach LU było dużo zer;

% TODO: Przykład strzałki Wilkinsona.

% Metody iteracyjne / stacjonarne.
\subsection{Metody stacjonarne}

\entry
$A=M-Z$, $Ax=Mx-Zx=b$, $Mx=b+Zx$, $Mx_{k+1}=b+Zx_{k}$ (iteracja prosta);

\entry
Metoda (**): $x_{k+1}=M^{-1}(b+Zx_k)$;

\entry
Chcemy, aby $Mx_{k+1}=\tilde{b}$ było łatwe;

\entry
Tw.: $\norm{M^{-1}Z} < 1$, to $\forall x_0 \in \mathbb{R}^N$ metoda (**) zbieżna do rozw. $Ax=b$;

\entry
Tw.: $\forall x_0$ metoda (**) zbieżna do $x^* \leftrightarrow \rho(M^{-1}Z) < 1$, gdzie $\rho(B) \coloneqq \max\set{\abs{\lambda} : \lambda\text{---w. wł. } B}$;

% XXX: Da się więcej powiedzieć.
\entry
Metoda Jacobiego: $M=\mathrm{diag}(A) = D$; zbieżna, gdy $A$ ma dominująca przekątną, tzn. $\abs{a_{ii}} > \sum_{j\neq i}\abs{a_{ij}} \forall i=1\ldots N$;

% TODO: Metoda Gaussa-Seidla.

% TODO: Metody przestrzeni Kryłowa.

% TODO: Metoda gradientów sprzężonych (CG).
