\mnsection{Układy równań liniowych}

\entry
Metody podstawienia w przód/tył. Koszt: $\bigoh{2n^2}$;

% Rozwiązywanie układu z macierzą ortogonalną.
\entry
$Q^TQ=I$, $Qx=b$, to $Q^TQx=Q^Tb$, to $x=Q^Tb$ w $\bigoh{n^2}$;

% TODO: Układ blokowy [A1 E; F C] * [x; y] = [f; g].

% TODO: Układ, w którym znany jest rozkład macierzy.
% A=BC, gdzie B, C --- macierze łatwe. Wówczas By=b, Cx=y

% TODO: Algorytm rozkładu LU (eliminacji Gaussa z wyborem elementu głównego w kolumnie).

% Metoda eliminacji Gaussa.
\entry
\textbf{M. el. G. (LU)}, gdy $A$ nieos..
Do $L$ wpisuj odwrotne znaki.
$\bigoh{\frac{2}{3}n^3}$;

\entry
$A$ dodatnio określona $\iff \forall_{x \neq 0} x^TAx>0$;

% TODO: Zapis algorytmu.
\entry
\textbf{R. Chol.}:
$A=A^T>0$,
to
$A=LL^T$;
$\bigoh{\frac{1}{3}n^3}$;
$A=LDL^T$
nie używa
$\sqrt{\cdot}$;

% Losowe fakty na temat macierzy.
\entry
$AA^T$ jest symetryczna;
\entry
$AA^{-1}=I$, to $AA^T>0$;
% Wartości własne A są równe transpozycji A.
\entry
$\sigma(A) = \sigma(A^T)$;

\entry
$
A>0
\leftrightarrow A=A^Tx^TAx > 0
\leftrightarrow \sigma(A) > 0
\leftrightarrow AA^T
$;
