\section{Aproksymacja jednostajna}

% Aproksymacja jednostajna, norma supremum.
\entry
$\norm{g}_\infty \coloneqq \max_{x \in [a;b]} \abs{g(x)}$;

% Formuła trójczłonowa na wielomiany Czebyszewa.
% XXX: Czy jest poprawna?
\entry
Formuła trójczłonowa:
$T_{k + 1} = 2T_1(x)T_k(x) - T_{k-1}(x)$,
$T_1(x) = \cos(1 \cdot \theta)=x$,
$T_0(x) = 1$;

% Wniosek z formuły trójczłonowej.
\entry
$\abs{T_k(x)} \leq 1$
dla
$\abs{x} \leq 1$;

% Fakt o miejscach zerowych k-tego wielomianu Czebyszewa.
\entry
Miejsca zerowe $k$-tego wielomianu Czebyszewa są rzeczywiste, jednokrotne i w przedziale $(-1;1)$:
$x_j \coloneqq \cos(\frac{2j-1}{2k}\pi)$, 
$j=1,\ldots,k$;

% Fakt o lokalnych ekstremach k-tego wielomianu Czebyszewa.
\entry
Lokalne ekstrema $T_k$ w przedziale $[-1;1]$ są dane wzorem:
$y_j \coloneqq \cos \frac{j\pi}{k}$,
$j=1,\ldots,k$.
Pondato
$T_k(y_j) = (-1)^j$;

% Twierdzenie o własności minimaksu.
\entry
Tw. (własność minimaksu):
Spośród wszystkich wielomianów stopnia $k$ postaci
$1 \cdot x^k + o(x^k)$,
$\frac{1}{2^{k-1}} T_k(x)$
ma najmniejszą normę maksimum na [-1;1].
Tzn.
$\max_{x \in [-1;1]} \abs{\frac{1}{2^{k-1}} T_k(x)} \leq 
\max_{x \in [-1;1]} \abs{w(x)} 
\forall w(x) = x^k + o(x^k)$;

% Węzły Czebyszewa w interpolacji wielomianowej Lagrange'a.
\entry
$\norm{f-w}_{\infty, [a,b]} \leq 
\frac{\norm{f^{(n+1)}}_{\infty, [a,b]}}{(n+1)!} 
\norm{(x-x_0) \cdots (x - x_n)}_\infty$, 
gdzie
$\max_{x\in[a,b]} (x-x_0)\cdots(x-x_n)$
jest najmniejszy dla 
$[a,b] = [-1,1]$;
gdy $x_i$ to miejsca zerowe $T_{n + 1}(x)$.
% "Przesunięte" wielomiany Czebyszewa.
\entry
$T^{*}_n(x) = T_n(Ax+B), x \in [a;b], Ax+b \in [-1;1]$;

% Zadanie aproksymacji jednostajnej wielomianami.
\entry
$f \in C[a,b]$.
$\exists p^{*} \in \mathbb{P}_n$ 
najlepszej aproksymacji dla $f$ w sensie normy supremum na $[a,b]$, że
$\norm{f - p^{*}}_\infty \leq \norm{f - p}_\infty \forall p\in \mathbb{P}_n$;

% Twierdzenie Czebyszewa o alternansie
\entry
Tw. (Czebyszewa o alternansie):
$p^{*} \in \mathbb{P}_n$
jest ENA dla $f$
$\leftrightarrow$
istnieją co najmniej $n+2$ punkty $x_i$ w $[a,b]$, że
$f(x_i) - p^{*}(x_i) = \varepsilon \cdot (-1)^i \cdot \norm{f - p^{*}}_\infty$,
$\varepsilon \in \set{-1,+1}$;
($\abs{f(x_i) - p^{*}(x_i)} = \norm{f - p^{*}}_\infty$);

% Twierdzenie o jednoznaczności optymalnego wielomianowej aproksymacji jednostajnej.
\entry
Istnieje dokładnie 1 ENA dla zadanej 
$f\in C[a,b]$ 
w sensie aproksymacji jednostajnej;

% TODO: Wyznaczanie ENA dla wielomianowej jednostajnej nieskończonym algorytmem iteracyjnym Rameza.

% Fakt o błędzie najlepszej aproksymacji.
% TODO: Formuła barycentryczna na w(x).
\entry
$f \in C[-1,1]$,
$w$ --- w. i. L. dla $f$ opartej na w. Cz.:
$E_n \leq
\norm{f - w}_\infty \leq
(2 + 2 \ln(n + 1) / \pi)\cdot E_n <_{n \leq 10^5}
10\cdot E_n$;
