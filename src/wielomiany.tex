\mnsection{Wielomiany}

% Interpolacja Lagrange'a.
\entry
\textbf{I. L.}:
$w(x_i) = f(x_i)$;

% Interpolacja Hermite'a.
\entry
\textbf{I. Hermite'a}:
$\deriv{w}{k}(x_i) = \deriv{f}{k}_i, k = 0 \ldots m$;

% Twierdzenie o interpolacji.
\entry
Tw.:
$\ipoints[i=0][n][a=][=b]{x_i}, f(x_i)$,
to
$\exists!_{w \in \mathbb{P}_n} w(x_i)=f(x_i)$;

% Dowód/intuicja dla twierdzenia o interpolacji.
\entry
$w(x_j) = y_i = \sum_{i=0}^n\alpha_i\varphi_i(x_j) = y_i$;
% Baza naturalna i macierz Vandermonde'a.
\entry
Dla \textbf{bazy nat.}
$\varphi_j(x)=x^j$
mamy \textbf{m. Vandermonde'a}:
$[x^j_i]$
(gęsta, niesym., zwykle b. źle uwar.);

% Baza Newtona.
\entry
\textbf{B. Newtona}:
$\varphi_i(x) \coloneqq (x-x_0)\cdots(x-x_i),
(w_n(x)-w_{n-1}(x) = b_n\varphi_n(x))$;

% Algorytm różnic dzielonych.
% (MP: 30 XI/3/1332)
\entry
\textbf{A. różnic dzielonych)}:
$
f[x_0] = f(x_0),
f[x_0,\ldots,x_k] = (f[x_1,\ldots,x_k] - f[x_0,\ldots,x_{k-1}])/(x_k - x_0),
f[x_1, x_1, x_1] = \frac{f''(x_1)}{2!}
$;

% TODO: Zmodyfikowany algorytm Hornera dla wielomianów w bazie Newtona.
% (MP: 30 XI/4/134)
% (MP: 1 XII/1.5/3)

% Twierdzenie o błędzie interpolacji.
\textbf{Tw. o bł. i.}:
$f\in C^{n+1}[a;b], \ipoints[0][n][a=][=b]{x_i}, w_n(x)$ jest w. i. L.
Wtedy
$\forallin{x}{[a;b]} \existsin{\xi}{(a;b)} f(x) - w_n(x) = \frac{\deriv{f}{n+1}(\xi)}{(n+1)!}\omega_n(x)$,
gdzie $\omega_n(x)=\varphi_{n+1}(x)$;

% TODO: Interpolacja Hermite'a.
% (MP: 30 XI/4.5/15)

% Baza Lagrange'a.
\entry
B. Lagrange'a:
$
l_i(x) = \frac{(x-x_0)\cdots(x-x_n)}{(x_i - x_0)\cdots(x_i - x_{i-1})(x_i - x_{i+1}) \cdots (x_i - x_n)},
l_i(x_j) = [i=j],
w(x_j)=f(x_j)\cdot 1
$;

% Algorytm barycentryczny dla wyznaczania wartości wielomianu interpolacyjnego Lagrange'a w bazie Lagrange'a.
\entry
A. barycentryczny dla w. i. L. w b. L.:
$\frac{w_i}{x - x_i}\varphi_{n+1}(x) \coloneqq l_i(x)$:
\textsubentry{1}
Wyznacz
$\{w_i\}$;
\textsubentry{2}
$w(x) = (\sum f(x_i)\frac{w_i}{x - x_i})\varphi_{n+1}(x)$;
$\mathcal{O}(n)$;

% Błąd interpolacji dla równoodległych węzłów.
% (MP: 1 XII/3/5)
\entry
Błąd i. dla $=_h$ węzłów:
$f \in C^{n+1}[a;b], x_i=a+ih$, to
$\max_{x \in [a;b]}\shortabs{f(x)-w(x)} \leq \frac{\max_\xi\shortabs{\deriv{f}{n+1}(\xi)} h^{n+1}}{4(n+1)}$,
$\shortabs{\varphi_{n+1}(x)} \leq \frac{1}{4} h^{n+1} n!$;
