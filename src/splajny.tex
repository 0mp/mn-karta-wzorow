\mnsection{Splajny}

% Definicja splajnu.
\entry
\textbf{Splajn}:
$\ipoints[0][n][a=][=b]{x_i}, s$
st. $l$ na $[a;b]$, gdy:
\textsubentry{1}
$s\in C^{l-1}[a;b]$;
\textsubentry{2}
$s\mid_{[x_i;x_{i+1}]} \in \mathbb{P}_l$;

% Fakt o przestrzenia splajnów.
% (MP: 7 XII/1.5/23)
\entry
Pń s. $S_l$ st. $l$ na $\ipoints{x_i}$ jest pnią liniową,
a jej bazą funkcje
$\set{p_0, \ldots, p_l, \varphi_1, \ldots, \varphi_{n-1}}$,
gdzie $p_i(x)=x^i$;
% TODO: Potwierdź prawdziwość poniższego stwierdzenia.
% (MP: 7 XII/1.5/231)
\entry
Wymiar tej pni to $(n+l)$;

% Zadanie interpolacji splajnowej.
\entry
\textbf{Zad. i. s.}:
$f$: znajdź $s\in S_l$, że
$s(x_i)=f(x_i), i=0\,\dots,n$;
