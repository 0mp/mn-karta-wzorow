\mnsection{Splajny}

% Definicja splajnu.
\entry
\textbf{Splajn}:
$\ipoints[0][n][a=][=b]{x_i}, s$
st. $l$ na $[a;b]$, gdy:
\textsubentry{1}
$s\in C^{l-1}[a;b]$;
\textsubentry{2}
$s\mid_{[x_i;x_{i+1}]} \in \mathbb{P}_l$;

% Fakt o przestrzenia splajnów.
% (MP: 7 XII/1.5/23)
\entry
Pń s. $S_l$ st. $l$ na $\ipoints{x_i}$ jest pnią liniową,
a jej bazą funkcje
$\set{p_0, \ldots, p_l, \varphi_1, \ldots, \varphi_{n-1}}$,
gdzie $p_i(x)=x^i$;
% TODO: Potwierdź prawdziwość poniższego stwierdzenia.
% (MP: 7 XII/1.5/231)
\entry
Wymiar tej pni to $(n+l)$;

% Zadanie interpolacji splajnowej.
\entry
\textbf{Zad. i. s.}:
$f$: znajdź $s\in S_l$, że
$s(x_i)=f(x_i), i=0\,\dots,n$;

% Warunki brzegowe.
% (MP: 7 XII/2/243)
\entry
\textbf{Warunki brzegowe}:
$l=2m-1 \Rightarrow 2m-2$ dodatkowych warunków;
% Hermitowskie warunki brzegowe.
\entry
\textbf{Hermitowskie}:
$\deriv{s}{k}(a) = \deriv{f}{k}(a), \ipoints[k=1][m-1]{k}$;
% Naturalne warunki brzegowe.
% (MP: 14 XII/1/112)
\entry
\textbf{Naturalne}:
$\deriv{s}{k}(a)=\deriv{s}{k}(b)=0, \ipoints[k=m][2m-2]{k}$;
% Periodyczne warunki brzegowe.
% (MP: 14 XII/1/113)
\entry
Periodyczne:
$\deriv{s}{k}(a)=\deriv{s}{k}(b), \ipoints[k=1][2m-2]{k}$;

% Warunki narzucone na splajn.
% (MP: 7 XII/2.5/2451)
\entry
War. narzucone na s.:
\textsubentry{1}
Interpolacja:
$s_i(x_i)=f(x_i)=a_i$;
\textsubentry{2}
Ciągłość II pochodnej:
$s_i''(x) = 2c_i+bd_i(x-x_i) \Rightarrow d_i=\frac{c_{i+1} - c_i}{3h}, \ipoints[0][n-2]{i}$;
\textsubentry{3}
Ciągłość s. w $x_{i+1}$:
$s_i(x_{i+1}) = s_{i+1}(x_{i+1}) \Rightarrow b_i = \frac{y_{i+1} - y_i}{h} - \frac{h}{3}(c_{i+1} + 2c_i)$;
\textsubentry{4}
Ciągłość pochodnej w $x_{i+1}$:
$
s_i'(x) = b_i +2c_i(x - x_i) + 3 d_i (x - x_i)^2,
s_i'(x_{i+1}) = s_{i+1}'(x_{x+1}),
b_{i+1} = b_i + 2 c_i h + 3 d_i h^2
$;

% Wzór na zależność pomiędzy c_i a y_i.
\entry
Z warunków n. n. s.:
$
g_{i+1} \coloneqq
c_i + 4c_{i+1} + c_{i+2} =
\frac{3}{h}(\frac{y_{i+2} - y_{i+1}}{h} - \frac{y_{i+1} - y_{i}}{h}),
i=0,\ldots, n-2,
\mathrm{tridiag}[\prescript{4}{1}{}^{1}] \cdot [c_i]^{n-1}_{1} = [g_i]^{n-1}_{1}
$

% Twierdzenie o aproksymacji naturalnym splajnem kubicznym interpolującym $f \in C^2[a;b]$.
\entry
\textbf{Aproks. nat. s. \mancube}:
$\ipoints[0][n][a=][=b]{a+ih}$:
$\norm{f-s}_\infty \leq c\norm{f''}_\infty h^2 = \mathcal{O}(h^2)$;

% Twierdzenie o błędzie aproksymacji interpolacyjnym hermitowskim splajnem kubicznym.
% (MP: 14 XII/1.5/12)
\entry
\textbf{Aproks. herm. s. \mancube}:
$\ipoints[0][n][a=][=b]{a+ih}$
(dla uproszczenia),
$
f\in C^{4}[a;b]
\Rightarrow
\norm{f-s}_\infty \leq \frac{5}{324} h^4 \norm{\deriv{f}{4}}_\infty
\wedge
\norm{f'-s'}_\infty \leq \frac{1}{24} h^3 \norm{\deriv{f}{4}}_\infty
$;

% B-splajny.

% Definicja nośnika.
% (MP: 14 XII/2/2112)
\entry
Nośnik:
$\mathrm{supp}(f) \coloneqq \set{x \in \mathbb{R}: f(x) \neq 0}$;

% Definicja B-splajnów.
\entry
\textbf{B-splajn}:
$B_i^k(x) \coloneqq V_i^k(x) B_i^{k-1}(x) + (1 - V^k_{i+1}(x)) B^{k-1}{i+1}$,
gdzie
$V_i^k(x) \coloneqq \frac{x-x_i}{x_{i+k} - x_i}$;
