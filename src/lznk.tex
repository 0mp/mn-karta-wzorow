\section{LZNK}

% TODO: Opisać algorytm iteracyjnego poprawiania w modelowej sytuacji.

\entry
Jeżeli $\mathrm{cond}(A)$ jest niezbyt duże względem $\frac{1}{\nu}$, to modelowy algorytm IR jest zbieżny;

\entry
LZNK: układ $n$ równań z $m$ niewiadomymi.
Cel: $\norm{b - Ax}_2 \to \min !$;
% $\sum_i(\sum_ja_{ij}-b_i)^2 \to \min !$

% TODO: Dodać rysunki macierzy.
\entry
Rozkład QR:
$A = QR = Q_1R_1$;

\entry
Algorytm rozwiązywania LZNK przez rozkład QR:
\textsubentry{1}
$A=QR$
\textsubentry{2}
$\norm{b-Ax}^2_2 = \norm{Q^T_1b-R_1x}^2_2 + \norm{\tilde{Q}^Tb}^2_2 \to \min !$
\textsubentry{3}
Rozwiązaniem jest $x\in \mathbb{R}^m$ spełniający ukł. r-ń z macierzą trójkątną $R_1x=Q_1^Tb$.
Koszt: $O(m^2)$.
Koszt $Q_1^Tb$: $O(nm)$;

% TODO: Wyznaczanie rozkładu QR metodą Householdera.
% TODO: Definicja przekształcenia Householdera.
% TODO: Odbicie Householdera.

\entry
Koszt rozkładu QR met. Householdera: $T(n,m)=2m^2n - \frac{2}{3}m^3$;
