\documentclass[10pt,a4paper,twocolumn]{article}

\usepackage{multicol}
\usepackage{amsbsy, amssymb, latexsym, amsmath, braket}
\usepackage[tiny]{titlesec}
\usepackage[hmargin=0.5cm,vmargin=1.0cm]{geometry}
\usepackage[utf8x]{inputenc}
\usepackage{polski}
\usepackage{scalefnt}
\usepackage[yyyymmdd,hhmmss]{datetime}
\usepackage{commath}

\newcommand{\entry}{$\bullet$\hspace{0.15em}}
\newcommand{\subentry}{$\circledcirc$\hspace{0.15em}}
% https://tex.stackexchange.com/a/7045/80219
\newcommand{\textsubentry}[1]{\tikz[baseline=(char.base)]{
            \node[shape=circle,draw,inner sep=1pt] (char) {#1};\hspace{0.15em}}}

\titlespacing{\section}{0pt}{0pt}{0pt}
\titlespacing{\subsection}{0pt}{0pt}{0pt}
\titlespacing{\subsubsection}{0pt}{0pt}{0pt}

% Wyłącz numerowanie stron.
\pagenumbering{gobble}

\setlength{\parindent}{0pt}
% Odległość pomiędzy liniami. Zmniejsz, jeżeli brakuje miejsca.
\setlength{\parskip}{0.5ex}

\title{Karta wzorów z metod numerycznych}

\begin{document}
% Rozmiar czcionki.
\scalefont{.8}

\text{\tiny{
    Wersja z \today\ o \currenttime\ (\pdfmdfivesum file{./karta-wzorow.tex})
}}

\section{Układy równań liniowych}

% TODO: Wzory kolejne x_k.
\entry
Metody podstawienia w przód/tył. Koszt: $O(2n^2)$;

% Rozwiązywanie układu z macierzą ortogonalną.
\entry
$Q^TQ=I$, $Qx=b$, to $Q^TQx=Q^Tb$, to $x=Q^Tb$ w $O(n^2)$;

% TODO: Układ blokowy [A1 E; F C] * [x; y] = [f; g].

% TODO: Układ, w którym znany jest rozkład macierzy.
% A=BC, gdzie B, C --- macierze łatwe. Wówczas By=b, Cx=y

% TODO: Algorytm rozkładu LU (eliminacji Gaussa z wyborem elementu głównego w kolumnie).
\entry
Metoda el. Gaussa (LU), gdy $A$ nieosobliwa. Koszt: $O(\frac{2}{3}n^3)$;

\entry
$A$ dodatnio określona $\iff \forall_{x \neq 0} x^TAx>0$;

% TODO: Zapis algorytmu.
\entry
Rozkład Choleskiego: $A=A^T>0$, to $A=LL^T$. Koszt: $O(\frac{1}{3}n^3)$. Rozkład $A=LDL^T$ nie używa pierwiastkowania;


\end{document}
