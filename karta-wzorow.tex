\documentclass[10pt,a4paper,twocolumn]{article}

\usepackage{multicol}
\usepackage{amsbsy, amssymb, latexsym, amsmath, braket}
\usepackage[tiny]{titlesec}
\usepackage[hmargin=0.5cm,vmargin=1.0cm]{geometry}
\usepackage[utf8x]{inputenc}
\usepackage{polski}
\usepackage{scalefnt}
\usepackage[yyyymmdd,hhmmss]{datetime}
\usepackage{commath}
\usepackage{mathtools}

% Definicje.
% https://tex.stackexchange.com/a/138901/80219
\newcommand{\Hsquare}{%
  \text{\fboxsep=-.2pt\fbox{\rule{0pt}{1ex}\rule{1ex}{0pt}}}%
}

\newcommand{\entry}{$\bullet$\hspace{0.15em}}
\newcommand{\subentry}{$\circledcirc$\hspace{0.15em}}
% https://tex.stackexchange.com/a/7045/80219
\newcommand{\textsubentry}[1]{\tikz[baseline=(char.base)]{
            \node[shape=circle,draw,inner sep=1pt] (char) {#1};\hspace{0.15em}}}

\titlespacing{\section}{0pt}{0pt}{0pt}
\titlespacing{\subsection}{0pt}{0pt}{0pt}
\titlespacing{\subsubsection}{0pt}{0pt}{0pt}

% Wyłącz numerowanie stron.
\pagenumbering{gobble}

\setlength{\parindent}{0pt}
% Odległość pomiędzy liniami. Zmniejsz, jeżeli brakuje miejsca.
\setlength{\parskip}{0.5ex}

\title{Karta wzorów z metod numerycznych}

\begin{document}
% Rozmiar czcionki.
\scalefont{.8}

\text{\tiny{
    Wersja z \today\ o \currenttime\ (\pdfmdfivesum file{./karta-wzorow.tex})
}}

\mnsection{Układy równań liniowych}

\entry
Metody podstawienia w przód/tył. Koszt: $O(2n^2)$;

% Rozwiązywanie układu z macierzą ortogonalną.
\entry
$Q^TQ=I$, $Qx=b$, to $Q^TQx=Q^Tb$, to $x=Q^Tb$ w $O(n^2)$;

% TODO: Układ blokowy [A1 E; F C] * [x; y] = [f; g].

% TODO: Układ, w którym znany jest rozkład macierzy.
% A=BC, gdzie B, C --- macierze łatwe. Wówczas By=b, Cx=y

% TODO: Algorytm rozkładu LU (eliminacji Gaussa z wyborem elementu głównego w kolumnie).
\entry
Metoda el. Gaussa (LU), gdy $A$ nieosobliwa. Koszt: $O(\frac{2}{3}n^3)$;

\entry
$A$ dodatnio określona $\iff \forall_{x \neq 0} x^TAx>0$;

% TODO: Zapis algorytmu.
\entry
R. Chol.:
$A=A^T>0$,
to
$A=LL^T$;
$O(\frac{1}{3}n^3)$;
$A=LDL^T$
nie używa
$\sqrt{\cdot}$;

% Losowe fakty na temat macierzy.
\entry
$AA^T$ jest symetryczna;
\entry
$A$ odwracalna, to $AA^T>0$;

\section{Arytmetyka fl}

% TODO: Dokładniejszy opis.
\entry
Liczby maszynowe: $x=(-1)^s \cdot m \cdot \beta^e$, gdzie $0\leq m < \beta$ i $m=(f_0, \ldots, f_{p-1})_\beta$, $f_i\in\set{0,1,\ldots,\beta-1}$, gdzie $p$ to precyzja arytmetyki;

\entry
Liczby znormalizowane: $x=(-1)^s\cdot m \cdot 2^e$, gdzie $1\leq m < 2$;

\entry
Gdy $x$ l. maszynową, to $\mathrm{fl}(x)=x$, wpp. $\mathrm{fl}(x)=\mathrm{ROUND}(x)=\mathrm{RN}(x)$, gdzie $\mathrm{RN}(x)$ zaokrągla do najbliższej l. maszynowej;

\entry
Błąd reprezentacji: $\abs{\mathrm{fl}(x) - x}$;

\entry
$\frac{\abs{\mathrm{fl}(x) - x}}{\abs{x}}\leq 2^{-p}=\nu$, o ile nie ma *flow;

\entry
$\mathrm{fl}(x) = x(1+\epsilon_x)$, $\abs{\epsilon_x}\leq \nu$

\entry
Standard fl gwarantuje, że $a \Hsquare b = \mathrm{fl}(a\Hsquare b)$, jeżeli nie ma NaN, *flow;

\entry
Fused multiply-add:
$\mathrm{fl}(\mathrm{FMA}(a,b,c)) = \mathrm{fl}(a \cdot b + c)$;

% Błędy w obliczeniach numerycznych.

\entry
Błąd bezwzględny $=\norm{\tilde{x} - x}$;

\entry
Błąd względny $=\frac{\text{błąd bezwzgl.}}{\norm{x}}$;

\entry
Silna numeryczna poprawność (NP): algorytm $A$ jest silnie numerycznie poprawny (NP), jeśli $\tilde{y} = P(\tilde{x})$, gdzie $\norm{\tilde{x}-x} / x\leq k\cdot\nu$, $P$---zadanie, $\tilde{y}$---dokładne rozw. dla trochę zaburzonych danych;

\entry
Słaba numeryczna poprawność (stabilność): daje prawie dokł. rozw. dla prawie dokł. danych: $\norm{\tilde{x}-x} / x\leq k\cdot\nu$, $\norm{\tilde{y}-P(\tilde{x})} / P(\tilde{x})\leq k\cdot\nu$;

\section{Normy}

% Normy wektorowe w $\mathbb{R}^n$.

\entry
$\norm{x}_1 \coloneqq \sum^n_{i=1}\abs{x_i}$;
\entry
$\norm{x}_2 \coloneqq \sqrt{x_1^2+\ldots + x_n^2}$;

\entry
$\norm{x}_\infty \coloneqq \max_{i}\abs{x_i}$;
\entry
$\norm{x}_p \coloneqq (\sum_i\abs{x_i}^p)^{1/p}$;

% Nierówność Cauchy’ego-Schwarza.
\entry
$\abs{x^Ty}_2 \leq \norm{x}_2\cdot\norm{y}_2$, dla $x,y\in\mathbb{R^N}$;

\entry
$Q^TQ=I$, to $\norm{Qx}^2_2=\norm{x}^2_2$;

% Normy macierzowe indukowane przez normy wektorowe.

\entry
$\norm{A}_p \coloneqq \max\limits_{x\neq 0} \frac{\norm{A_x}_p}{\norm{x}_p} = \max\limits_{\norm{x}_p = 1} \norm{A_x}_p$;

% Fakty.
\entry
$\norm{I} = 1$;
\entry
$\norm{Ax} \leq \norm{A} \cdot \abs{x}$;
\entry
$\norm{AB} \leq \norm{A} \cdot \norm{B}$;

\entry
$\norm{A}_1 \coloneqq \max_j\sum_i\abs{a_{ij}}$;
\entry
$\norm{A}_\infty \coloneqq \max_i\sum_j\abs{a_{ij}}$;

\entry
$\norm{A}_2 \coloneqq \max\set{\sqrt{\lambda} : \lambda\text{ --- wartość własna } A^TA}$;

\entry
$\norm{D} = \max_i d_i$;

\entry
Jeśli $\forall_{x} \norm{Ax}\leq c\norm{x}$, że $\exists_y \norm{Ay} = c\norm{y}$, to $c=\norm{A}$;

% Fakt z zadania 2.4. z części 2 ćwiczeń Konrada Sakowskiego.
\entry
$\norm{A}_2 = \sup_{\norm{x}_2=1, \norm{y}_2=1}\abs{y^TAx}$;

% Fakt z zadania 2.5. z części 2 ćwiczeń Konrada Sakowskiego.
\entry
$A\in\mathbb{R}^{N\times N} \Rightarrow \norm{A}^2_2=\norm{A^TA}_2$;

% Normy macierzowe nieindukowane przez normy wektorowe.

\entry
Norma Frobeniusa: $\norm{A}_F \coloneqq \sqrt{\sum_i\sum_j\abs{a_{ij}}^2}$; $\norm{I}_F=\sqrt{N}$;




\end{document}
