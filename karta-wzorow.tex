\documentclass[10pt,a4paper,twocolumn]{article}

\usepackage{multicol}
\usepackage{amsbsy, amssymb, latexsym, amsmath, braket}
\usepackage[tiny]{titlesec}
\usepackage[hmargin=0.5cm,vmargin=1.0cm]{geometry}
\usepackage[utf8x]{inputenc}
\usepackage{polski}
\usepackage{scalefnt}
\usepackage[yyyymmdd,hhmmss]{datetime}
\usepackage{commath}
\usepackage{mathtools}

% Definicje.
% https://tex.stackexchange.com/a/138901/80219
\newcommand{\Hsquare}{%
  \text{\fboxsep=-.2pt\fbox{\rule{0pt}{1ex}\rule{1ex}{0pt}}}%
}

\newcommand{\entry}{$\bullet$\hspace{0.15em}}
\newcommand{\subentry}{$\circledcirc$\hspace{0.15em}}
% https://tex.stackexchange.com/a/7045/80219
\newcommand{\textsubentry}[1]{\tikz[baseline=(char.base)]{
            \node[shape=circle,draw,inner sep=1pt] (char) {#1};\hspace{0.15em}}}

\titlespacing{\section}{0pt}{0pt}{0pt}
\titlespacing{\subsection}{0pt}{0pt}{0pt}
\titlespacing{\subsubsection}{0pt}{0pt}{0pt}

% Wyłącz numerowanie stron.
\pagenumbering{gobble}

\setlength{\parindent}{0pt}
% Odległość pomiędzy liniami. Zmniejsz, jeżeli brakuje miejsca.
\setlength{\parskip}{0.5ex}

\title{Karta wzorów z metod numerycznych}

\begin{document}
% Rozmiar czcionki.
\scalefont{.8}

\text{\tiny{
    Wersja z \today\ o \currenttime\ (\pdfmdfivesum file{./karta-wzorow.tex})
}}

\section{Układy równań liniowych}

% TODO: Wzory kolejne x_k.
\entry
Metody podstawienia w przód/tył. Koszt: $O(2n^2)$;

% Rozwiązywanie układu z macierzą ortogonalną.
\entry
$Q^TQ=I$, $Qx=b$, to $Q^TQx=Q^Tb$, to $x=Q^Tb$ w $O(n^2)$;

% TODO: Układ blokowy [A1 E; F C] * [x; y] = [f; g].

% TODO: Układ, w którym znany jest rozkład macierzy.
% A=BC, gdzie B, C --- macierze łatwe. Wówczas By=b, Cx=y

% TODO: Algorytm rozkładu LU (eliminacji Gaussa z wyborem elementu głównego w kolumnie).
\entry
Metoda el. Gaussa (LU), gdy $A$ nieosobliwa. Koszt: $O(\frac{2}{3}n^3)$;

\entry
$A$ dodatnio określona $\iff \forall_{x \neq 0} x^TAx>0$;

% TODO: Zapis algorytmu.
\entry
Rozkład Choleskiego: $A=A^T>0$, to $A=LL^T$. Koszt: $O(\frac{1}{3}n^3)$. Rozkład $A=LDL^T$ nie używa pierwiastkowania;

\mnsection{Arytmetyka fl}

% TODO: Dokładniejszy opis.
\entry
Liczby maszynowe: $x=(-1)^s \cdot m \cdot \beta^e$, gdzie $0\leq m < \beta$ i $m=(f_0, \ldots, f_{p-1})_\beta$, $f_i\in\set{0,1,\ldots,\beta-1}$, gdzie $p$ to precyzja arytmetyki;

\entry
Liczby znormalizowane: $x=(-1)^s\cdot m \cdot 2^e$, gdzie $1\leq m < 2$;

\entry
$x$ to l. m.,
to $\mathrm{fl}(x)=x$,
wpp. $\mathrm{fl}(x)=\mathrm{ROUND}(x)=\mathrm{RN}(x)$;
% gdzie $\mathrm{RN}(x)$ zaokrągla do najbl. l. m.;

\entry
$\mathrm{fl}(x) = x(1+\varepsilon_x)$, $\abs{\varepsilon_x}\leq \nu$
\entry
Błąd reprezentacji: $\shortabs{\mathrm{fl}(x) - x}$;

\entry
$\flatfrac{\shortabs{\mathrm{fl}(x) - x}}{\shortabs{x}}\leq 2^{-p}=\nu$, o ile nie ma *flow;
% Fused multiply-add.
\entry
$\mathrm{fl}(\mathrm{FMA}(a,b,c)) = \mathrm{fl}(a \cdot b + c)$;

\entry
Standard fl gwarantuje, że $a\, \Hsquare\, b = \mathrm{fl}(a \,\Hsquare\, b)$, jeżeli nie ma NaN i *flowów;

% Błędy w obliczeniach numerycznych.

\entry
\textbf{Błąd bezwzględny}:
$=\norm{\tilde{x} - x}$;
\entry
\textbf{Błąd względny}:
$=\frac{\text{błąd bezwzgl.}}{\norm{x}}$;

\entry
\textbf{Silna numeryczna poprawność} (NP):
alg. $A$ jest s. n. p.,
jeśli $\tilde{y} = P(\tilde{x})$,
gdzie $\norm{\tilde{x}-x} / x\leq k\cdot\nu$,
$P$---zadanie,
$\tilde{y}$---dokł. rozw. dla trochę zab. danych;

\entry
\textbf{Słaba numeryczna poprawność} (stabilność):
daje prawie dokł. rozw. dla prawie dokł. danych:
$\norm{\tilde{x}-x} / x\leq k\cdot\nu$, $\norm{\tilde{y}-P(\tilde{x})} / P(\tilde{x})\leq k\cdot\nu$;

\section{Normy}

% Normy wektorowe w $\mathbb{R}^n$.

\entry
$\norm{x}_1 \coloneqq \sum^n_{i=1}\abs{x_i}$;
\entry
$\norm{x}_2 \coloneqq \sqrt{x_1^2+\ldots + x_n^2}$;

\entry
$\norm{x}_\infty \coloneqq \max_{i}\abs{x_i}$;
\entry
$\norm{x}_p \coloneqq (\sum_i\abs{x_i}^p)^{1/p}$;

\entry
$\abs{x^Ty}_2=\norm{x}_2\cdot\norm{y}_2$, dla $x,y\in\mathbb{R^N}$;

\entry
$Q^TQ=I$, to $\norm{Qx}^2_2=\norm{x}^2_2$;

% Normy macierzowe indukowane przez normy wektorowe.

\entry
$\norm{A}_p \coloneqq \max\limits_{x\neq 0} \frac{\norm{A_x}_p}{\norm{x}_p} = \max\limits_{\norm{x}_p = 1} \norm{A_x}_p$;

% Fakty.
\entry
$\norm{I} = 1$;
\entry
$\norm{Ax} \leq \norm{A} \cdot \abs{x}$;
\entry
$\norm{AB} \leq \norm{A} \cdot \norm{B}$;

\entry
$\norm{A}_1 \coloneqq \max_j\sum_i\abs{a_{ij}}$;
\entry
$\norm{A}_\infty \coloneqq \max_i\sum_j\abs{a_{ij}}$;

\entry
$\norm{A}_2 \coloneqq \max\set{\sqrt{\lambda} : \lambda\text{ --- wartość własna } A^TA}$;

\entry
$\norm{D} = \max_i d_i$;

\entry
Jeśli $\forall_{x} \norm{Ax}\leq c\norm{x}$, że $\exists_y \norm{Ay} = c\norm{y}$, to $c=\norm{A}$;

% Normy macierzowe nieindukowane przez normy wektorowe.

\entry
Norma Frobeniusa: $\norm{A}_F \coloneqq \sqrt{\sum_i\sum_j\abs{a_{ij}}^2}$; $\norm{I}_F=\sqrt{N}$;



\mnsection{Uwarunkowanie zadania}

\entry
Wskaźnik u. $P$ w punkcie $x$:
$\mathrm{cond}_{\mathrm{abs}}(P,x) \coloneqq \sup\limits_{\text{małe }\delta} \frac{\norm{P(x+\delta) - P(x)}}{\norm{\delta}}$;

\entry
$\norm{P(x+\delta)} \leq \mathrm{cond}_{\mathrm{abs}}(P,x)\norm{\delta}$;

\entry
$\frac{\norm{P(x+\delta) - P(x)}}{\norm{P(x)}} \leq \mathrm{cond}_{\mathrm{rel}}(P,x)\frac{\norm{\delta}}{\norm{x}}$;

% Idealizacja.

\entry
$\mathrm{cond}_{\mathrm{abs}}(P,x) \coloneqq \lim\limits_{\norm{\delta \to 0}} \frac{\norm{P(x+\delta) - P(x)}}{\norm{\delta}}$;

\entry
$\mathrm{cond}_{\mathrm{rel}}(P,x) \coloneqq \mathrm{cond}_{\mathrm{abs}}(P,x)\frac{\norm{x}}{\norm{P(x)}} $;

\entry
Zadanie $P$ jest źle uwarunkowane w punkcie $x$,
gdy $\mathrm{cond}(P,x) \gg 1$,
bo małe zaburzenie danych może spowodować duży błąd wyniku;

% XXX: Czy to powinny być normy drugie?
\entry
$\mathrm{cond}(A) \coloneqq \norm{\vphantom{A^{-1}}A}\cdot\norm{A^{-1}}$;

\entry
$A=A^T\in\mathbb{R}^{N\times N}$,
to
$\mathrm{cond}_2(A)\coloneqq \frac{\max_i\abs{\lambda_i}}{\min_i\abs{\lambda_i}}$
i
$\exists_{v_i\neq 0} Av_i=\lambda_i v_i$;

\entry
$\exists_{v_i\neq 0} A_{v_i} = \lambda_i v_i$,
gdzie $v_i$ --- wektor własny;

\entry
Gdy
$\varepsilon\cdot\mathrm{cond}(A) < \frac{1}{2}$,
to $\frac{\norm{x - \tilde{x}}}{\norm{x}} \leq 4 \cdot \mathrm{cond}(A) \cdot \varepsilon$;

\entry
Jeśli $\norm{B} < 1$,
to $I+B$ odwracalna i $\norm{(I+B)^{-1}} \leq 1 / (1 - \norm{B})$;

\entry
Jeśli $A$ odwracalna i $\norm{A^{-1}\Delta} < 1$,
to $(A+\Delta)^{-1}$ istnieje
i $\norm{(A+\Delta)^{-1}} \leq \frac{\norm{A^{-1}}}{1 - \norm{A^{-1}} \cdot \norm{\vphantom{A^1}\Delta}}$,
gdzie $\Delta \leq \varepsilon\norm{A}$;

% Fakt o rozwiązaniu \tilde{x} układu Ax=b za pomocą LU z wyborem w kolumnie.
\entry
M. e. G. dla $Ax=b$ daje $\tilde{x}$ t., że
$(A+\Delta) \tilde{x} = b$,
gdzie $\frac{\norm{\Delta}_\infty}{\norm{A}_\infty} \leq K\cdot N^3 \cdot \rho_N \cdot \nu$,
gdzie $\rho_N \coloneqq \frac{\max_{i,j,k}\abs{a_{i,j}^{(k)}}}{\max_{i,j} \abs{a_{i,j}}}$,
gdzie $a^{(k)}_{i,j}$ to wyraz macierzy w $k$-tym kroku LU. W praktyce wskaźnik wzrostu $\rho_N\leq 2^{N-1}$, więc m. el. G. jest prak. NP.

% Numeryczne kryterium NP.
\entry
Numeryczne kryterium NP:
gdy $\tilde{x}$ przybliżonym rozw. $Ax=b$,\\
to $(A+\Delta)\tilde{x}=b+\delta$
i $\frac{\norm{\delta}}{\norm{b}}, \frac{\norm{\Delta}}{\norm{A}} \leq \varepsilon$,
gdzie $\varepsilon \coloneqq \frac{\norm{b-A\tilde{x}}}{\norm{A} \cdot \norm{\tilde{x}} + \norm{b} }$;

% TODO: Przykłady zadań dobrze i źle uwarunkowanych.

% Lemat.
\entry
$\norm{\Delta} < 1$,
to $I + \Delta$ nieosobliwa
oraz $\norm{(I + \Delta)^{-1}} \leq \frac{1}{1 - \norm{\Delta}}$;

\section{Macierze rzadkie}

% Metody bezpośrednie.
\entry
Metody bezpośrednie:
r. LU z permutacją wierszy i kolumn, aby w czynnikach LU było dużo zer;

% TODO: Przykład strzałki Wilkinsona.

% Metody iteracyjne / stacjonarne.
\subsection{Metody stacjonarne}

% Ogólna idea metod stacjonarnych (1).
\entry
$A=M-Z$;
\entry
$Mx=b+Zx$;
\entry
$Mx_{k+1}=b+Zx_{k}$ (iteracja prosta);
\entry
$x_{k+1}=x_k+M^{-1}r-k$, gdzie $r_k=b-Ax_k$;

% Ogólna idea metod stacjonarnych (2).
\entry
Metoda: $\mathrm{(**)}$ $x_{k+1}=M^{-1}(b+Zx_k)$,
aby $Mx_{k+1}=\tilde{b}$ było łatwe;

% Twierdzenie o zbieżności metody iteracyjnej (1).
\entry
Tw.: $\norm{M^{-1}Z} < 1$,
to metoda $\mathrm{(**)}$ zbieżna do rozw. $Ax=b \forall x_0 \in \mathbb{R}^N$

% Twierdzenie o zbieżności metody iteracyjnej (2).
\entry
Tw.: $\forall x_0$,
metoda $\mathrm{(**)}$
zbieżna do $x^* \leftrightarrow \rho(M^{-1}Z) < 1$,
gdzie $\rho(B) \coloneqq \max\set{\abs{\lambda} : \lambda\text{ --- w. wł. } B}$;

% TODO: Dodaj przykładowy algorytm implementujący iterację.

% Metoda Jacobiego.
\entry
Metoda Jacobiego:
$M=\mathrm{diag}(A) = D$, $A=L+D+U$;
zbieżna, gdy $A$ ma dominująca przekątną,
tzn. $\abs{a_{ii}} > \sum_{j\neq i}\abs{a_{ij}} \forall i=1, \ldots, N$;

% Metoda Gaussa-Seidla.
\entry
Metoda Gaussa-Seidla:
$M = L + D$;
zbieżna, gdy $a_{ii}x_i^{k+1}=b_i-\sum_{j>i}a_{ij}x^k-\sum_{j<i}a_{ij}x^{k+1}$;

% Metoda Richardsona.
\entry
Metoda Richardsona:
$x_k+1 = x_k + \tau(b-Ax_k)$, gdzie $\tau\in\mathbb{R}$.
Gdy $A=A^T>0$,
to $\tau_{\mathrm{opt}}=\frac{2}{\lambda_{\mathrm{min}} + \lambda_{\mathrm{max}} }$
i $\gamma \coloneqq \norm{M^{-1}Z}=\abs{1-\lambda_{\mathrm{max}}\tau_{\mathrm{opt}}}=\frac{\mathrm{cond}_2(A)-1}{\mathrm{cond}_2(A) + 1}$;

\subsection{Metody przestrzeni Kryłowa}

% Definicja iteracji w metodach przestrzeni Kryłowa.
\entry
$k$-ta iteracja:
$x_k\in x_o+K_k$,
gdzie $K_k\coloneqq \set{r_0, Ar_0,\ldots, A^{k-1}r_0}$,
gdzie $r_0=b-Ax_0$;

% TODO: Coś tu jest nie tak.
% Metody oparte na minimalizacji błędu.
\entry
Metody oparte na minimalizacji błędu:
$x_k\in x_0 +K_k$ oraz $\norm{x_k - x^*}_B \forall x\in x_0 + K_k$;

% Metody oparte na minimalizacji residuum.
\entry
Metody oparte na minimalizacji residuum:
$\norm{b-Ax_k}_B \leq \norm{b-Ax}_B \forall x\in x_0 +K_k$, gdzie $B=B^T>0$;

% Metoda gradientów sprzężonych (CG).
\entry
Metoda gradientów sprzężonych (CG):
$A=A^T>0$, $x_k\in x_0 + K_k$ t.,
że $\norm{x_k+x^*}_A\leq \norm{x-x^*}_A \forall x\in x_0 + K_k$,
gdzie $\norm{y}^2_A \coloneqq y^TAy$.
Koszt iteracji, to mnożenie wektora przez $A$, czyli $O(N)$.
W idealnej arytmetyce zbieżne do $x^*$ w $\leq N$ iteracjach.
Po $k$ iteracjach:
$\norm{x_k - x^*}_A\leq 2(\frac{\sqrt{H} - 1}{\sqrt{H} + 1})^k\norm{x_0-x^*}_A$,
gdzie $H=\mathrm{cond}_2(A)$;

% TODO: Wzmianka o GMRES.


\end{document}
