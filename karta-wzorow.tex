\documentclass[10pt,a4paper,twocolumn]{article}

\usepackage{multicol}
\usepackage{amsbsy, amssymb, latexsym, amsmath, braket}
\usepackage[tiny]{titlesec}
\usepackage[hmargin=0.5cm,vmargin=1.0cm]{geometry}
\usepackage[utf8x]{inputenc}
\usepackage{polski}
\usepackage{scalefnt}
\usepackage[yyyymmdd,hhmmss]{datetime}
\usepackage{commath}
\usepackage{mathtools}
\usepackage{tikz}
\usepackage{xparse}
\usetikzlibrary{tikzmark}

% Definicje.
% https://tex.stackexchange.com/a/138901/80219
\newcommand{\Hsquare}{%
  \text{\fboxsep=-.2pt\fbox{\rule{0pt}{1ex}\rule{1ex}{0pt}}}%
}

% Hack, który sprawia, że \cdot w wyrażeniu \norm{A}\cdot\norm{A} znajduje się
% w równej odległości od obu norm. W przeciwnym wypadku \cdot znajduje się
% dziwnie blisko prawej normy.
% https://tex.stackexchange.com/a/99882/80219
% https://tex.stackexchange.com/a/61511/80219
\DeclareRobustCommand*{\norm}[1]{\|#1\|}

% Symbol "jak wyżej".
% https://tex.stackexchange.com/a/53844
\newcommand{\dittotikz}{%
    \tikz{
        % Pionowe kreski.
        \draw [line width=0.12ex] (-0.2ex,0) -- +(0,0.8ex)
            (0.2ex,0) -- +(0,0.8ex);
        % Poziome kreski.
        \draw [line width=0.08ex] (-0.6ex,0.4ex) -- +(-0.8em,0)
            (0.6ex,0.4ex) -- +(0.8em,0);
    }%
}

% Makra pomocnicze.
% Element najlepszej aproksymacji (ENA), zazwyczaj wyróżniany gwiazdką
% w górnym indeksie.
\newcommand{\ena}[1]{#1^{*}}
\newcommand{\enain}[2]{#1^{*} \in #2}
\newcommand{\forallin}[2]{\forall_{#1 \in #2}}
\newcommand{\existsin}[2]{\exists_{#1 \in #2}}
\newcommand{\maxin}[2]{\max_{#1 \in #2}}

% Krótkie \abs. Przydatne, gdy standardowe \abs wygląda niezgrabnie.
\newcommand{\shortabs}[1]{\mid\! #1\! \mid}

% Uwarunkowanie. Opcjonalny argument służy do podania normy.
% Na przykład: $\cond[2]{A}$.
\newcommand{\cond}[2][]{\mathrm{cond}_{#1}(#2)}

% Unormowanie wektora.
% Argumenty:
%   * wektor
%   * oznaczenie normy (opcjonalnie)
%   * potęga przy normie (opcjonalnie)
% Przykład: $\normify[\infty][2]{x_{k+1}}$
\DeclareDocumentCommand{\normify}{ O{} O{} m}{#3 = #3 / \norm{#3}_{#1}^{#2}}
\DeclareDocumentCommand{\normifyfrac}{ O{} O{} m}{#3 = \frac{#3}{ \norm{#3}_{#1}^{#2}}}

% Przydate do pętli iterujących nieznaną liczbę razy.
% Obowiązkowy argument to iterator, a opcjonalny argument to jego wartość
% początkowa domyślnie równa 0.
% Na przykład: $\fromloop[1]{k}$
\newcommand{\fromloop}[2][0]{\mathrm{from}(#2\!\!=\!\!#1)}

% Pochodna.
% Na przykład: $\deriv{w}{k}$
\newcommand{\deriv}[2]{#1^{(#2)}}

% Zbiór węzłów interpolacyjncyh.
% Indeksy po lewej: https://tex.stackexchange.com/a/64612/80219
% Na przykład: $\ipoints[i=0][n][a][b]{x_i}$
\DeclareDocumentCommand{\ipoints}{O{} O{} O{} O{} m}{\prescript{#4}{#3}{\{#5\}}^{#2}_{#1}}

\newcommand{\entry}{$\bullet$\hspace{0.15em}}
\newcommand{\subentry}{$\circledcirc$\hspace{0.15em}}
% https://tex.stackexchange.com/a/7045/80219
\newcommand{\textsubentry}[1]{\tikz[baseline=(char.base)]{
            \node[shape=circle,draw,inner sep=1pt] (char) {#1};\hspace{0.15em}}}

\newcommand{\mnsection}[1]{

    \hrulefill
    \textbf{{#1}}
    \hrulefill
}

% Wyłącz numerowanie stron.
\pagenumbering{gobble}

\setlength{\parindent}{0pt}
% Odległość pomiędzy liniami. Zmniejsz, jeżeli brakuje miejsca.
\setlength{\parskip}{0.5ex}

\title{Karta wzorów z metod numerycznych}

\begin{document}
% Rozmiar czcionki.
\scalefont{.8}

\text{\tiny{
    Wersja
    \input{|"git rev-list --count --first-parent HEAD"}z
    \today\ o \currenttime\ (\pdfmdfivesum file{./karta-wzorow.tex})
}}

\section{Układy równań liniowych}

% TODO: Wzory kolejne x_k.
\entry
Metody podstawienia w przód/tył. Koszt: $O(2n^2)$;

% Rozwiązywanie układu z macierzą ortogonalną.
\entry
$Q^TQ=I$, $Qx=b$, to $Q^TQx=Q^Tb$, to $x=Q^Tb$ w $O(n^2)$;

% TODO: Układ blokowy [A1 E; F C] * [x; y] = [f; g].

% TODO: Układ, w którym znany jest rozkład macierzy.
% A=BC, gdzie B, C --- macierze łatwe. Wówczas By=b, Cx=y

% TODO: Algorytm rozkładu LU (eliminacji Gaussa z wyborem elementu głównego w kolumnie).
\entry
Metoda el. Gaussa (LU), gdy $A$ nieosobliwa. Koszt: $O(\frac{2}{3}n^3)$;

\entry
$A$ dodatnio określona $\iff \forall_{x \neq 0} x^TAx>0$;

% TODO: Zapis algorytmu.
\entry
Rozkład Choleskiego: $A=A^T>0$, to $A=LL^T$. Koszt: $O(\frac{1}{3}n^3)$. Rozkład $A=LDL^T$ nie używa pierwiastkowania;

\mnsection{Arytmetyka fl}

% TODO: Dokładniejszy opis.
\entry
Liczby maszynowe: $x=(-1)^s \cdot m \cdot \beta^e$, gdzie $0\leq m < \beta$ i $m=(f_0, \ldots, f_{p-1})_\beta$, $f_i\in\set{0,1,\ldots,\beta-1}$, gdzie $p$ to precyzja arytmetyki;

\entry
Liczby znormalizowane: $x=(-1)^s\cdot m \cdot 2^e$, gdzie $1\leq m < 2$;

\entry
$x$ to l. m.,
to $\mathrm{fl}(x)=x$,
wpp. $\mathrm{fl}(x)=\mathrm{ROUND}(x)=\mathrm{RN}(x)$;
% gdzie $\mathrm{RN}(x)$ zaokrągla do najbl. l. m.;

\entry
$\mathrm{fl}(x) = x(1+\varepsilon_x)$, $\abs{\varepsilon_x}\leq \nu$
\entry
Błąd reprezentacji: $\shortabs{\mathrm{fl}(x) - x}$;

\entry
$\flatfrac{\shortabs{\mathrm{fl}(x) - x}}{\shortabs{x}}\leq 2^{-p}=\nu$, o ile nie ma *flow;
% Fused multiply-add.
\entry
$\mathrm{fl}(\mathrm{FMA}(a,b,c)) = \mathrm{fl}(a \cdot b + c)$;

\entry
Standard fl gwarantuje, że $a\, \Hsquare\, b = \mathrm{fl}(a \,\Hsquare\, b)$, jeżeli nie ma NaN i *flowów;

% Błędy w obliczeniach numerycznych.

\entry
\textbf{Błąd bezwzględny}:
$=\norm{\tilde{x} - x}$;
\entry
\textbf{Błąd względny}:
$=\frac{\text{błąd bezwzgl.}}{\norm{x}}$;

\entry
\textbf{Silna numeryczna poprawność} (NP):
alg. $A$ jest s. n. p.,
jeśli $\tilde{y} = P(\tilde{x})$,
gdzie $\norm{\tilde{x}-x} / x\leq k\cdot\nu$,
$P$---zadanie,
$\tilde{y}$---dokł. rozw. dla trochę zab. danych;

\entry
\textbf{Słaba numeryczna poprawność} (stabilność):
daje prawie dokł. rozw. dla prawie dokł. danych:
$\norm{\tilde{x}-x} / x\leq k\cdot\nu$, $\norm{\tilde{y}-P(\tilde{x})} / P(\tilde{x})\leq k\cdot\nu$;

\section{Normy}

% Normy wektorowe w $\mathbb{R}^n$.

\entry
$\norm{x}_1 \coloneqq \sum^n_{i=1}\abs{x_i}$;
\entry
$\norm{x}_2 \coloneqq \sqrt{x_1^2+\ldots + x_n^2}$;

\entry
$\norm{x}_\infty \coloneqq \max_{i}\abs{x_i}$;
\entry
$\norm{x}_p \coloneqq (\sum_i\abs{x_i}^p)^{1/p}$;

\entry
$\abs{x^Ty}_2=\norm{x}_2\cdot\norm{y}_2$, dla $x,y\in\mathbb{R^N}$;

\entry
$Q^TQ=I$, to $\norm{Qx}^2_2=\norm{x}^2_2$;

% Normy macierzowe indukowane przez normy wektorowe.

\entry
$\norm{A}_p \coloneqq \max\limits_{x\neq 0} \frac{\norm{A_x}_p}{\norm{x}_p} = \max\limits_{\norm{x}_p = 1} \norm{A_x}_p$;

% Fakty.
\entry
$\norm{I} = 1$;
\entry
$\norm{Ax} \leq \norm{A} \cdot \abs{x}$;
\entry
$\norm{AB} \leq \norm{A} \cdot \norm{B}$;

\entry
$\norm{A}_1 \coloneqq \max_j\sum_i\abs{a_{ij}}$;
\entry
$\norm{A}_\infty \coloneqq \max_i\sum_j\abs{a_{ij}}$;

\entry
$\norm{A}_2 \coloneqq \max\set{\sqrt{\lambda} : \lambda\text{ --- wartość własna } A^TA}$;

\entry
$\norm{D} = \max_i d_i$;

\entry
Jeśli $\forall_{x} \norm{Ax}\leq c\norm{x}$, że $\exists_y \norm{Ay} = c\norm{y}$, to $c=\norm{A}$;

% Normy macierzowe nieindukowane przez normy wektorowe.

\entry
Norma Frobeniusa: $\norm{A}_F \coloneqq \sqrt{\sum_i\sum_j\abs{a_{ij}}^2}$; $\norm{I}_F=\sqrt{N}$;



\mnsection{Uwarunkowanie zadania}

\entry
Wskaźnik u. $P$ w punkcie $x$:
$\mathrm{cond}_{\mathrm{abs}}(P,x) \coloneqq \sup\limits_{\text{małe }\delta} \frac{\norm{P(x+\delta) - P(x)}}{\norm{\delta}}$;

\entry
$\norm{P(x+\delta)} \leq \mathrm{cond}_{\mathrm{abs}}(P,x)\norm{\delta}$;

\entry
$\frac{\norm{P(x+\delta) - P(x)}}{\norm{P(x)}} \leq \mathrm{cond}_{\mathrm{rel}}(P,x)\frac{\norm{\delta}}{\norm{x}}$;

% Idealizacja.

\entry
$\mathrm{cond}_{\mathrm{abs}}(P,x) \coloneqq \lim\limits_{\norm{\delta \to 0}} \frac{\norm{P(x+\delta) - P(x)}}{\norm{\delta}}$;

\entry
$\mathrm{cond}_{\mathrm{rel}}(P,x) \coloneqq \mathrm{cond}_{\mathrm{abs}}(P,x)\frac{\norm{x}}{\norm{P(x)}} $;

\entry
Zadanie $P$ jest źle uwarunkowane w punkcie $x$,
gdy $\mathrm{cond}(P,x) \gg 1$,
bo małe zaburzenie danych może spowodować duży błąd wyniku;

% XXX: Czy to powinny być normy drugie?
\entry
$\mathrm{cond}(A) \coloneqq \norm{\vphantom{A^{-1}}A}\cdot\norm{A^{-1}}$;

\entry
$A=A^T\in\mathbb{R}^{N\times N}$,
to
$\mathrm{cond}_2(A)\coloneqq \frac{\max_i\abs{\lambda_i}}{\min_i\abs{\lambda_i}}$
i
$\exists_{v_i\neq 0} Av_i=\lambda_i v_i$;

\entry
$\exists_{v_i\neq 0} A_{v_i} = \lambda_i v_i$,
gdzie $v_i$ --- wektor własny;

\entry
Gdy
$\varepsilon\cdot\mathrm{cond}(A) < \frac{1}{2}$,
to $\frac{\norm{x - \tilde{x}}}{\norm{x}} \leq 4 \cdot \mathrm{cond}(A) \cdot \varepsilon$;

\entry
Jeśli $\norm{B} < 1$,
to $I+B$ odwracalna i $\norm{(I+B)^{-1}} \leq 1 / (1 - \norm{B})$;

\entry
Jeśli $A$ odwracalna i $\norm{A^{-1}\Delta} < 1$,
to $(A+\Delta)^{-1}$ istnieje
i $\norm{(A+\Delta)^{-1}} \leq \frac{\norm{A^{-1}}}{1 - \norm{A^{-1}} \cdot \norm{\vphantom{A^1}\Delta}}$,
gdzie $\Delta \leq \varepsilon\norm{A}$;

% Fakt o rozwiązaniu \tilde{x} układu Ax=b za pomocą LU z wyborem w kolumnie.
\entry
M. e. G. dla $Ax=b$ daje $\tilde{x}$ t., że
$(A+\Delta) \tilde{x} = b$,
gdzie $\frac{\norm{\Delta}_\infty}{\norm{A}_\infty} \leq K\cdot N^3 \cdot \rho_N \cdot \nu$,
gdzie $\rho_N \coloneqq \frac{\max_{i,j,k}\abs{a_{i,j}^{(k)}}}{\max_{i,j} \abs{a_{i,j}}}$,
gdzie $a^{(k)}_{i,j}$ to wyraz macierzy w $k$-tym kroku LU. W praktyce wskaźnik wzrostu $\rho_N\leq 2^{N-1}$, więc m. el. G. jest prak. NP.

% Numeryczne kryterium NP.
\entry
Numeryczne kryterium NP:
gdy $\tilde{x}$ przybliżonym rozw. $Ax=b$,\\
to $(A+\Delta)\tilde{x}=b+\delta$
i $\frac{\norm{\delta}}{\norm{b}}, \frac{\norm{\Delta}}{\norm{A}} \leq \varepsilon$,
gdzie $\varepsilon \coloneqq \frac{\norm{b-A\tilde{x}}}{\norm{A} \cdot \norm{\tilde{x}} + \norm{b} }$;

% TODO: Przykłady zadań dobrze i źle uwarunkowanych.

% Lemat.
\entry
$\norm{\Delta} < 1$,
to $I + \Delta$ nieosobliwa
oraz $\norm{(I + \Delta)^{-1}} \leq \frac{1}{1 - \norm{\Delta}}$;

% \section{Macierze rzadkie}

% Metody bezpośrednie.
\entry
Metody bezpośrednie:
r. LU z permutacją wierszy i kolumn, aby w czynnikach LU było dużo zer;

% TODO: Przykład strzałki Wilkinsona.

% Metody iteracyjne / stacjonarne.
\subsection{Metody stacjonarne}

% Ogólna idea metod stacjonarnych (1).
\entry
$A=M-Z$;
\entry
$Mx=b+Zx$;
\entry
$Mx_{k+1}=b+Zx_{k}$ (iteracja prosta);
\entry
$x_{k+1}=x_k+M^{-1}r-k$, gdzie $r_k=b-Ax_k$;

% Ogólna idea metod stacjonarnych (2).
\entry
Metoda: $\mathrm{(**)}$ $x_{k+1}=M^{-1}(b+Zx_k)$,
aby $Mx_{k+1}=\tilde{b}$ było łatwe;

% Twierdzenie o zbieżności metody iteracyjnej (1).
\entry
Tw.: $\norm{M^{-1}Z} < 1$,
to metoda $\mathrm{(**)}$ zbieżna do rozw. $Ax=b \forall x_0 \in \mathbb{R}^N$

% Twierdzenie o zbieżności metody iteracyjnej (2).
\entry
Tw.: $\forall x_0$,
metoda $\mathrm{(**)}$
zbieżna do $x^* \leftrightarrow \rho(M^{-1}Z) < 1$,
gdzie $\rho(B) \coloneqq \max\set{\abs{\lambda} : \lambda\text{ --- w. wł. } B}$;

% TODO: Dodaj przykładowy algorytm implementujący iterację.

% Metoda Jacobiego.
\entry
Metoda Jacobiego:
$M=\mathrm{diag}(A) = D$, $A=L+D+U$;
zbieżna, gdy $A$ ma dominująca przekątną,
tzn. $\abs{a_{ii}} > \sum_{j\neq i}\abs{a_{ij}} \forall i=1, \ldots, N$;

% Metoda Gaussa-Seidla.
\entry
Metoda Gaussa-Seidla:
$M = L + D$;
zbieżna, gdy $a_{ii}x_i^{k+1}=b_i-\sum_{j>i}a_{ij}x^k-\sum_{j<i}a_{ij}x^{k+1}$;

% Metoda Richardsona.
\entry
Metoda Richardsona:
$x_k+1 = x_k + \tau(b-Ax_k)$, gdzie $\tau\in\mathbb{R}$.
Gdy $A=A^T>0$,
to $\tau_{\mathrm{opt}}=\frac{2}{\lambda_{\mathrm{min}} + \lambda_{\mathrm{max}} }$
i $\gamma \coloneqq \norm{M^{-1}Z}=\abs{1-\lambda_{\mathrm{max}}\tau_{\mathrm{opt}}}=\frac{\mathrm{cond}_2(A)-1}{\mathrm{cond}_2(A) + 1}$;

\subsection{Metody przestrzeni Kryłowa}

% Definicja iteracji w metodach przestrzeni Kryłowa.
\entry
$k$-ta iteracja:
$x_k\in x_o+K_k$,
gdzie $K_k\coloneqq \set{r_0, Ar_0,\ldots, A^{k-1}r_0}$,
gdzie $r_0=b-Ax_0$;

% TODO: Coś tu jest nie tak.
% Metody oparte na minimalizacji błędu.
\entry
Metody oparte na minimalizacji błędu:
$x_k\in x_0 +K_k$ oraz $\norm{x_k - x^*}_B \forall x\in x_0 + K_k$;

% Metody oparte na minimalizacji residuum.
\entry
Metody oparte na minimalizacji residuum:
$\norm{b-Ax_k}_B \leq \norm{b-Ax}_B \forall x\in x_0 +K_k$, gdzie $B=B^T>0$;

% Metoda gradientów sprzężonych (CG).
\entry
Metoda gradientów sprzężonych (CG):
$A=A^T>0$, $x_k\in x_0 + K_k$ t.,
że $\norm{x_k+x^*}_A\leq \norm{x-x^*}_A \forall x\in x_0 + K_k$,
gdzie $\norm{y}^2_A \coloneqq y^TAy$.
Koszt iteracji, to mnożenie wektora przez $A$, czyli $O(N)$.
W idealnej arytmetyce zbieżne do $x^*$ w $\leq N$ iteracjach.
Po $k$ iteracjach:
$\norm{x_k - x^*}_A\leq 2(\frac{\sqrt{H} - 1}{\sqrt{H} + 1})^k\norm{x_0-x^*}_A$,
gdzie $H=\mathrm{cond}_2(A)$;

% TODO: Wzmianka o GMRES.

% Metody iteracyjne / stacjonarne.
\section{Metody stacjonarne}

% Ogólna idea metod stacjonarnych (1).
\entry
$A=M-Z$;
\entry
$Mx_{k+1}=b+Zx_{k}$
\entry
$x_{k+1}=x_k+M^{-1}r_k$, gdzie $r_k=b-Ax_k$;

% Ogólna idea metod stacjonarnych (2).
\entry
Metoda: $\mathrm{(**)}$ $x_{k+1}=M^{-1}(b+Zx_k)$,
aby $Mx_{k+1}=\tilde{b}$ było łatwe;

% Twierdzenie o zbieżności metody iteracyjnej (1).
\entry
Tw.: $\norm{M^{-1}Z} < 1$,
to metoda $\mathrm{(**)}$ zbieżna do rozw. $Ax=b \forall x_0 \in \mathbb{R}^N$

% Twierdzenie o zbieżności metody iteracyjnej (2).
\entry
Tw.: $\forall x_0$,
metoda $\mathrm{(**)}$
zbieżna do $x^* \leftrightarrow \rho(M^{-1}Z) < 1$,
gdzie $\rho(B) \coloneqq \max\set{\abs{\lambda} : \lambda\text{ --- w. wł. } B}$;

% Przykładowy algorytm implementujący iterację.
\entry
$\mathrm{iter}(A,b,x_0,M,t)$:
$x=x_0;
r=b - Ax;
(L,U,P) = lu(M);
\mathrm{while}(\norm{r} > t) \{
    p = M^{-1}r;
    x=x+p;
    r=b-Ax
\};
\mathrm{return}(x)
$;
$O(N^3 + \mathrm{\#iter} \cdot N^2)$;

% Metoda Jacobiego.
\entry
Metoda Jacobiego:
$M=\mathrm{diag}(A) = D$, $A=L+D+U$;
zbieżna, gdy $A$ ma dominująca przekątną,
tzn. $\abs{a_{ii}} > \sum_{j\neq i}\abs{a_{ij}} \forall i=1, \ldots, N$;

% Metoda Gaussa-Seidla.
\entry
Metoda Gaussa-Seidla:
$M = L + D$;
zbieżna, gdy $a_{ii}x_i^{k+1}=b_i-\sum_{j>i}a_{ij}x^k-\sum_{j<i}a_{ij}x^{k+1}$;

% Metoda Richardsona.
\entry
Metoda Richardsona:
$x_k+1 = x_k + \tau(b-Ax_k)$, gdzie $\tau\in\mathbb{R}$.
Gdy $A=A^T>0$,
to $\tau_{\mathrm{opt}}=\frac{2}{\lambda_{\mathrm{min}} + \lambda_{\mathrm{max}} }$
i $\gamma \coloneqq \norm{M^{-1}Z}=\abs{1-\lambda_{\mathrm{max}}\tau_{\mathrm{opt}}}=\frac{\mathrm{cond}_2(A)-1}{\mathrm{cond}_2(A) + 1}$;

% Algorytm iteracyjnego poprawiania w modelowej sytuacji.
\entry
Modelowe IR:
$\mathrm{for}(n=0,\ldots)\{ r_n=b-Ax_n; Ae_n=_{\mathrm{fl}}r_n; x_{n+1} = x_n + e_n \}$;

% TODO: Dodać dowody zbieżności modelowego algorytmu.

% Zbieżność modelowego algorytmu iteracyjnego poprawiania rozwiązania (IR, czyli iterative reconstruction)
\entry
Gdy $\mathrm{cond}(A)$ jest niezbyt duże względem $\frac{1}{\nu}$, to modelowy IR jest zbieżny;


\mnsection{Metody przestrzeni Kryłowa}

% Definicja iteracji w metodach przestrzeni Kryłowa.
\entry
$k$-ta iteracja:
$x_k\in x_o+K_k$,
gdzie $K_k\coloneqq \set{r_0, Ar_0,\ldots, A^{k-1}r_0}$,
gdzie $r_0=b-Ax_0$;

% TODO: Coś tu jest nie tak.
% Metody oparte na minimalizacji błędu.
\entry
M. oparte na min.:
błędu:
$x_k\in x_0 +K_k$ oraz $\norm{x_k - x^*}_B \forall x\in x_0 + K_k$;
% Metody oparte na minimalizacji residuum.
residuum:
$\norm{b-Ax_k}_B \leq \norm{b-Ax}_B \forall x\in x_0 +K_k$, gdzie $B=B^T>0$;

% Metoda gradientów sprzężonych (CG).
\entry
Metoda gradientów sprzężonych (CG):
$A=A^T>0$, $x_k\in x_0 + K_k$ t.,
że $\norm{x_k+x^*}_A\leq \norm{x-x^*}_A \forall x\in x_0 + K_k$,
gdzie $\norm{y}^2_A \coloneqq y^TAy$.
Koszt iteracji, to mnożenie wektora przez $A$, czyli $\bigoh{N}$.
W idealnej arytmetyce zbieżne do $x^*$ w $\leq N$ iteracjach.
Po $k$ iteracjach:
$\norm{x_k - x^*}_A\leq 2(\frac{\sqrt{H} - 1}{\sqrt{H} + 1})^k\norm{x_0-x^*}_A$,
gdzie $H=\mathrm{cond}_2(A)$;

% TODO: Wzmianka o GMRES.

\section{LZNK}

% LZNK.
\entry
LZNK: układ $n$ równań z $m$ niewiadomymi.
Cel: $\norm{b - Ax}_2 \to \min !$;
% $\sum_i(\sum_ja_{ij}-b_i)^2 \to \min !$

% Układ równań normalnych.
\entry
U. normalny
$A^TAx=A^Tb$
jest rozw. LZNK;

% Uwarunkowanie LZNK.
\entry
Uwarunkowanie LZNK:
$\norm{\tilde{b} - \tilde{A}\tilde{x}}_2 \to \mathrm{min}!$
i $\epsilon\coloneqq \max\set{ \frac{\norm{A - \tilde{A}}_2}{\norm{A}_2}, \frac{\norm{x - \tilde{x}}_2}{\norm{x}_2} }$
tak małe, że $\mathrm{cond}_2(A)\cdot \epsilon < 1$,
gdzie $\mathrm{cond}_2(A) = \frac{\sigma_{\max(A)}}{\min(A)}$.
Wtedy zadanie ma 1 rozwiązanie i
$\frac{\norm{x-\tilde{x}}_2}{\norm{x}_2} \leq  \epsilon \cdot \mathrm{cond}_{\mathrm{LZNK}}(A,b)+ O(\epsilon^2)$,
gdzie $\mathrm{cond}_{\mathrm{LZNK}}(A,b) = \frac{2\mathrm{cond}_2(A)}{\cos\theta} + \tan\theta(\mathrm{cond}_2(A))^2$,
gdzie $\sin\theta=\frac{\norm{b-Ax}_2}{\norm{b}_2}$;

% Rozkład QR.
\entry
QR:
$A = QR = Q_1R_1$,
gdzie
$Q^TQ=Q^T_1Q_1=I$,
$Q_1\in\mathbb{R}^{n\times m}$,
$R_1\in\mathbb{R}^{m\times m}$;

% Algorytm rozwiązywania LZNK przez rozkład QR.
\entry
Rozw. LZNK przez r. QR:
\textsubentry{1}
$A=QR$;
\textsubentry{2}
$\norm{b-Ax}^2_2 = \norm{Q^T_1b-R_1x}^2_2 + \norm{\tilde{Q}^Tb}^2_2 \to \min !$;
\textsubentry{3}
Rozwiązaniem jest $x\in \mathbb{R}^m$ spełniający ukł. r-ń z macierzą trójkątną $R_1x=Q_1^Tb$.
Koszt: $O(m^2)$.
Koszt $Q_1^Tb$: $O(nm)$;

% TODO: Wyznaczanie rozkładu QR metodą Householdera.
% TODO: Definicja przekształcenia Householdera.

% Przekształcenie (metoda) Householdera.
\entry
Odbicie Householdera:
$v\in \mathbb{R}^N$,
$\gamma \coloneqq \frac{2}{\norm{v}^2_2} $,
$H\coloneqq I-\gamma vv^T$ ortogonalna, gdzie
$v$ postaci $a-ce_1$,
gdzie $a$ to zerowany wektor;

\entry
$H_m \cdots H_1 A = R$, $Q \coloneqq H_1 \cdots H_m$;
% Własności przekształcenia Householdera.
\entry
$H=H{-1}=H^T$;

\entry
Obliczanie $y=Qz$:
$y=z; \mathrm{for}(i=m,\ldots 1)\{y=H_iy\}$;

\entry
Koszt rozkładu QR met. Householdera: $T(n,m)=2m^2n - \frac{2}{3}m^3$;

\entry
Układ równań normalnych:
$A^TAx=A^Tb$;

% TODO: Obrót Givensa.

% TODO: Rozkład QR macierzy Hassenberga.

% Rozkład SVD.
\entry
Rozkład SVD:
$\forall A \in \mathbb{R}^{n\times m}, n \geq m, \exists A=U_1\Sigma_1V^T$, gdzie
$U_1^TU_1=V^TV=I$,
$\Sigma_1 = \mathrm{diag}(\sigma_i)$, gdzie
% TODO: Dodać brakujące założenia dotyczące wartości szczególnych.
$\sigma_i \geq 0$ to wart. szczególna $A$;

\entry
Fakt:
$A=A^T$, to mamy $A=U\Delta U^T$, gdzie $U$ to w. wł. $A$,
$\Delta= \mathrm{diag}(\lambda_i)$.
Wtedy SVD dla $A$ to $A=U\Sigma V^T$, gdzie
$\sigma_i=\abs{\lambda_i}, v_i=\mathrm{sign}(\lambda_i)u_i$;

\entry
Fakt:
W. wł. $A^TA$ to $\sigma_i^2,\ldots,\sigma_m^2$ i $(n-m)$ zer;

\entry
Tw.:
$A$ pełnego rzędu, to rozwiązaniem LZNK jest $x=V\Sigma_1^{-1}U_1^Tb$;

% TODO: Nieregularne zadanie LZNK dla rank(A) < m.

\mnsection{Zadanie własne}

% Metoda potęgowa wyznaczania wektora własnego.
% (9 XI/4.5/43)
\entry
M. pot.:
$\abs{\lambda_1}>\abs{\lambda_2}\geq\ldots\geq\abs{\lambda_n}$
$\Rightarrow$
$\mathrm{for}(k=0,\dots)\{x_{k+1}=Ax_k / \norm{x_{k+1}}^2_2\}$;

% Definicja zadania własnego.
\entry
Zadanie własne $A \in \mathbb{R}^{N \times N}$:
$(\lambda,x) \in \mathbb{C} \times \mathbb{C}^N$:
$Ax = \lambda x$;

% Fakt o wartościach własnych (1).
\entry
Gdy $\lambda$ dla $A$, to $(\lambda - \mu)$ dla $A-\mu I$;

% Fakt o wartościach własnych (2).
\entry
Gdy $\lambda$ dla $A$ nieos., to $\frac{1}{\lambda}$ dla $A^{-1}$;

% Wniosek z powyższych faktów dla metody potęgowej.
\entry
Met. potęgowa zastosowana do
$(A-\mu I)^{-1}$
będzie zbieżna do
$\lambda_i$,
czyli w. wł. najbliższej $\mu$.
Rzeczywiście, w. wł.
$(A-\mu I)^{-1}$,
to
$\frac{1}{\lambda_j-\mu}$
i jeśli
$\frac{1}{\abs{\lambda_i-\mu}} > \frac{1}{\abs{\lambda_j-\mu}}, j\neq i$,
to
$\frac{1}{\lambda_i-\mu}$
jest dominująca;

% ^^^^^^^^^^^^^^^^^^^^^^^^^^^^^
% Zakres materiału do kolokwium.

% Odwrotna metoda potęgowa.
% (MP: 16 XI/1.5/2)
\entry
\textbf{Odw. m. pot.}:
$\text{for } k=0,\ldots$:
$x_{k+1} = (A-\mu I)^{-1}x_k$;
$x_{k+1} = x_{k+1} / \norm{x_{k+1}}$;
$\mathcal{O}(n^3)$;
% Praktyczny algorytm implementujący odwrotna metodę potęgową.
\entry
Szybciej:
$P(A- \mu I = LU$ (lub $A-\mu I = QR$) ($\mathcal{O}{n^3}$);
$\text{for } k=0,\ldots$:
$Ly=Px_k$;
$Ux_{k+1} = y$;
$x_{k+1} = x_{k+1} / \norm{x_{k+1}}$;
($\mathcal{O}(kN^2)$)

% Najlepsze przybliżenie wartości własnej z użyciem ilorazu Rayleigh.
% (MP: 16 XI/1.5/23)
\entry
Znając przybliżony $x_k$ wek. wł.: mamy najlepsze (w sensie
średniokwadratowym) przybliżenie war. wł. (\textbf{iloraz Rayleigh}):
$\lambda_k \coloneqq x_k^T A x_k / x_k^T x_k$.
Gdy $x_k = v$:
$v^T A v / v^T v = v^T \lambda v / v^T v = \lambda$;

% Metoda Rayleigh (RQI).
% (MP: 16 XI/2/24)
\entry
\textbf{M. Rayleigh} (RQI):
$\text{for } k=0,\ldots$:
$x_{k+1} = (A - \mu_k I)^{-1} x_k$;
$x_{k+1} = x_{k+1} / \norm{x_{k+1}}$;
$\mu_{k+1} = x_{k+1}^T A x_{k+1}$.
Zbiega $\cdot^3$, a coraz gorsze uwarunkowanie ($A-\mu_k I$) pomaga.
Po $3$ iteracjach precyzja arytmetyki.

% TODO: Wartości własne macierzy blokowo-trójkątnej.
% (MP: 16 XI/2.5/3)

% Fakty o lokalizacji wartości własnych.
% (MP: 16 XI/3/4)
\entry
F. o lok. war. wł.:
$\abs{\lambda} \leq \norm{A}$;
% Definicja \sigma(A).
\entry
Zb. wartości wł. $A$:
$\sigma(A)$;

% Twierdzenie Gerszgorina.
\entry
Tw. Gerszgorina:
$\sigma(A) \in  \sum$ kół:
$K_i \coloneqq \set{z\in\mathbb{C}: \shortabs{z - a_{ii}} \leq \sum_{j \neq i}\shortabs{a_{ij}}}$;

% Twierdzenie (Bauer-Fike).
% (MP: 16 X/4/5)
\entry
Tw. (Bauer-Fike):
$A$ diagonalizowalna
($\exists_{X\text{ nieosobliwa}}$: $\Lambda \coloneqq X^{-1}AX=\mathrm{\lambda_i}_1^N$),
$\tilde{\lambda}$ war. wł. $\tilde{A} \coloneqq A + \Delta$.
Wtedy
$\min_{\lambda \in \sigma(A)} \shortabs{\lambda - \tilde{\lambda}} \leq
\cond[p]{X} \cdot \norm{\Delta}_p$;
% Wniosek z twierdzenia Bauera-Fike'a.
% (MP: 16 X/4.5/53)
% TODO: Dowód tego wniosku też jest ciekawy.
\entry
$A$ symetryczna:
$\min_i\shortabs{\lambda_i - \tilde{\lambda}_i} \leq \norm{\Delta}_2$;

% Metoda QR wyznaczania wszystkich wartości własnych macierzy (wraz z ulepszeniem i wariantem praktycznym).
\entry
M. QR na $\sigma(A)$:
$ A_1 \coloneqq A;
\fromloop[1]{k}\{
Q_k R_k \coloneqq A_k;
A_{k+1} \coloneqq R_k Q_k
\} $;
% Metoda QR z przesunięciem.
\entry
Lepiej:
$ A_1 \coloneqq A;
\fromloop[1]{k}\{
\text{wybierz przesunięcie } \sigma_k;
Q_k R_k \coloneqq A_k - \sigma_kI;
A_{k+1} \coloneqq R_k Q_k + \sigma_kI
\} $;
% Praktyczne QR.
\entry
Przed iteracją sprowadź $A$ do postaci Hassenberga lub trójdiagonalnej, gdy $A=A^T$;

\mnsection{Wielomiany}

% Interpolacja Lagrange'a.
\entry
\textbf{I. L.}:
$w(x_i) = f(x_i)$;

% Interpolacja Hermite'a.
\entry
\textbf{I. Hermite'a}:
$\deriv{w}{k}(x_i) = \deriv{f}{k}_i, k = 0 \ldots m$;

% Twierdzenie o interpolacji.
\entry
Tw.:
$\ipoints[i=0][n][a=][=b]{x_i}, f(x_i)$,
to
$\exists!_{w \in \mathbb{P}_n} w(x_i)=f(x_i)$;

% Dowód/intuicja dla twierdzenia o interpolacji.
\entry
$w(x_j) = y_i = \sum_{i=0}^n\alpha_i\varphi_i(x_j) = y_i$;
% Baza naturalna i macierz Vandermonde'a.
\entry
Dla \textbf{bazy nat.}
$\varphi_j(x)=x^j$
mamy \textbf{m. Vandermonde'a}:
$[x^j_i]$
(gęsta, niesym., zwykle b. źle uwar.);

% Baza Newtona.
\entry
\textbf{B. Newtona}:
$\varphi_i(x) \coloneqq (x-x_0)\cdots(x-x_i),
(w_n(x)-w_{n-1}(x) = b_n\varphi_n(x))$;

% Algorytm różnic dzielonych.
% (MP: 30 XI/3/1332)
\entry
\textbf{A. różnic dzielonych)}:
$
f[x_0] = f(x_0),
f[x_0,\ldots,x_k] = (f[x_1,\ldots,x_k] - f[x_0,\ldots,x_{k-1}])/(x_k - x_0),
f[x_1, x_1, x_1] = \frac{f''(x_1)}{2!}
$;

% TODO: Zmodyfikowany algorytm Hornera dla wielomianów w bazie Newtona.
% (MP: 30 XI/4/134)
% (MP: 1 XII/1.5/3)

% Twierdzenie o błędzie interpolacji.
\textbf{Tw. o bł. i.}:
$f\in C^{n+1}[a;b], \ipoints[0][n][a=][=b]{x_i}, w_n(x)$ jest w. i. L.
Wtedy
$\forallin{x}{[a;b]} \existsin{\xi}{(a;b)} f(x) - w_n(x) = \frac{\deriv{f}{n+1}(\xi)}{(n+1)!}\omega_n(x)$,
gdzie $\omega_n(x)=\varphi_{n+1}(x)$;

% TODO: Interpolacja Hermite'a.
% (MP: 30 XI/4.5/15)

% Baza Lagrange'a.
\entry
B. Lagrange'a:
$
l_i(x) = \frac{(x-x_0)\cdots(x-x_n)}{(x_i - x_0)\cdots(x_i - x_{i-1})(x_i - x_{i+1}) \cdots (x_i - x_n)},
l_i(x_j) = [i=j],
w(x_j)=f(x_j)\cdot 1
$;

% Algorytm barycentryczny dla wyznaczania wartości wielomianu interpolacyjnego Lagrange'a w bazie Lagrange'a.
\entry
A. barycentryczny dla w. i. L. w b. L.:
$\frac{w_i}{x - x_i}\varphi_{n+1}(x) \coloneqq l_i(x)$:
\textsubentry{1}
Wyznacz
$\{w_i\}$;
\textsubentry{2}
$w(x) = (\sum f(x_i)\frac{w_i}{x - x_i})\varphi_{n+1}(x)$;
$\mathcal{O}(n)$;

% Błąd interpolacji dla równoodległych węzłów.
% (MP: 1 XII/3/5)
\entry
Błąd i. dla $=_h$ węzłów:
$f \in C^{n+1}[a;b], \ipoints[i=0][n][a=][=b]{x_i=a+ih}$, to
$\max_{x \in [a;b]}\shortabs{f(x)-w(x)} \leq \frac{\max_\xi\shortabs{\deriv{f}{n+1}(\xi)} h^{n+1}}{4(n+1)}$,
$\shortabs{\varphi_{n+1}(x)} \leq \frac{1}{4} h^{n+1} n!$;

% Funkcja Rungego przykładem słabości interpolacji wielomianowej.
% (MP: 7 XII/1/1)
\entry
Interpolacja w. f. Rungego
$f(x)=1/(1+x^2)$
ma duży błąd.

\mnsection{Splajny}

% Definicja splajnu.
\entry
\textbf{Splajn}:
$\ipoints[0][n][a=][=b]{x_i}, s$
st. $l$ na $[a;b]$, gdy:
\textsubentry{1}
$s\in C^{l-1}[a;b]$;
\textsubentry{2}
$s\mid_{[x_i;x_{i+1}]} \in \mathbb{P}_l$;

% Fakt o przestrzenia splajnów.
% (MP: 7 XII/1.5/23)
\entry
Pń s. $S_l$ st. $l$ na $\ipoints{x_i}$ jest pnią liniową,
a jej bazą funkcje
$\set{p_0, \ldots, p_l, \varphi_1, \ldots, \varphi_{n-1}}$,
gdzie $p_i(x)=x^i$;
% TODO: Potwierdź prawdziwość poniższego stwierdzenia.
% (MP: 7 XII/1.5/231)
\entry
Wymiar tej pni to $(n+l)$;


\mnsection{Aproksymacja funkcji}

% Problem zadania najlepszej aproksymacji.
\entry
\textbf{ENA}:
$f \in F$
(pń unormowana).
Szukamy ENA
$\enain{v}{V \subset F}$ takiego, że
$\forallin{v}{V} \norm{f - \ena{v}} \leq \norm{f-v}$;

% Fakt o istnieniu ENA.
\entry
$F$ pnią lin.,
$V \subset F$ podpnią lin. skończonego wymiaru, to
$\forall{f \in F} \ \exists{\enain{v}{V}}$;

% Normy.
\entry
Normy:
\subentry
$L^1(a,b)$:
$\norm{f}_1 \coloneqq \int^b_a\abs{f(t)} dt$;
\subentry
$L^2(a,b)$:
$\norm{f}_2 \coloneqq (\int^b_a\abs{f(t)}^2 dt)^{1/2}$;
\subentry
$C^0(a,b)$:
$\norm{f}_\infty \coloneqq \maxin{t}{[a,b]} \abs{f(t)}$;
\subentry
$L^2_\rho(a,b)$:
$\norm{f}_{2, \rho} \coloneqq (\int^b_a\abs{f(t)}^2 \rho(t) \ dt)^{1/2}$,
gdzie
$\rho(t) > 0$ na $[a,b]$
(pewna skończona liczba punktów, gdzie się zeruje);

% TODO: Przykładowe własności powyższych norm.

\entry
$t_{\text{opt}}, t_v \in [a,b]$:
\subentry
$l_1$ (dyskretna)
$\sum^n_{i=0} \abs{f(t_i)}$;
\subentry
$l_2$ (\dittotikz)
$(\sum^n_{i=0} \abs{f(t_i)}^2)^{1/2}$;
\subentry
$l_{C_0}$ (\dittotikz)
$\max_{i = 0,\ldots,n} \abs{f(t_i)}$;
\subentry
$l_2$ (dyskretna ważona)
$(\sum^n_{i=0} \rho_i\abs{f(t_i)}^2)^{1/2}, \rho_i>0$;

% Aproksymacja w przestrzeni unitarnej.
\entry
$H$ to pń lin. z
$(x,y)$ dla $x, y \in H$
\subentry
$H \coloneqq \mathbb{R}^n$:
$(x,y) \coloneqq x^T y$;
\subentry
$H \coloneqq C^0[a;b]$:
$(f,g) \coloneqq \int^b_a f(t)g(t) \ dt$;
\subentry
$(\cdot, \cdot)$ indukuje normę w $H$:
$\norm{f} \coloneqq \sqrt{(f,f)}$;

% Twierdzenie o charakteryzacji ENA w przestrzeni unitarnej.
% TODO: Przykłady przestrzeni unitarnych (np. Hilberta).
\entry
\textbf{Tw. (o charakteryzacji ENA w pni unitarnej)}:
Niech $V \subset H$ będzie podpnią liniową w $H$:
$\enain{v}{V}$ jest ENA dla $f \in H$
$\leftrightarrow$
$(f - \ena{v}, v) = 0 \ \forallin{v}{V}$;

% Wniosek z twierdzenia o charakteryzacji ENA w przestrzeni unitarnej.
\entry
Jeśli $V$ jest skończonego wymiaru, to ENA istnieje i jest jedyny.
Gdy $\set{v_i}^n_{i=1}$ są bazą pni $V$, to
$\ena{v} = \sum^n_{i=1} \ena{\alpha}_i v_i$
i współczynniki $\set{\ena{\alpha}_i}$
spełniają układ równań normalnych:
$\sum \ena{\alpha}_i (v_i, v_j) = (f, v_j)$;

\entry
$[(v_i, v_j)]^n_{i,j=1}$
jest symetryczna i dodatnio określona (macierz Grama),
więc ma dokładnie 1 rozwiązanie;

% Macierz Grama.
\entry
\textbf{M. Grama}: zazwyczaj gęsta i źle uwar., SPD (więc alg. Chol.);

% Macierz Grama dla przestrzeni z bazą ortogonalną.
\entry
Wybierając bazę ortogonalną w $V = \text{span}\set{p_i}^n_{i=1}$,
gdzie $(p_i, p_j) = 0, i \neq j$, dostajemy macierz Grama
$[(p_i, p_i)]$,
$\ena{v} = \sum((f,p_i)\cdot p_i)/(p_i, p_i)$,
a gdy $\set{p_i}^n_{i=1}$ unormowane:
$\ena{v} = \sum(f,p_i)\cdot p_i$;

% Definicja wielomianów ortogonalnych dla L^2_\rho.
\entry
$L^2_\rho(a;b)$:
$(f,g) \coloneqq \int^b_a f(x)g(x)\rho(x) \ dx$,
$\rho$ to funkcja wagowa $\rho > 0$ na $[a;b]$
(w skończonej liczbie punktów może się zerować);
$\norm{f}^2_\rho \coloneqq (f,f)^{1/2}$;

% Formuła trójczłonowa.
\entry
\textbf{Formuła trójczłonowa}:
$P_k \in \mathbb{P}_k$ dla  $\{k\}_0$, postaci
$P_k = x^k + o(x^k)$ wyrażają się wzorem:
$P_k(x) = (x + \beta_k) \cdot P_{k-1}(x) + \gamma_k P_{k-2}(x)$
dla $\{k\}_1$, gdzie
$
P_{-1}(x)=0,
P_{0}(x)=1,
\beta_k= \frac{(x P_{k-1}, P_{k-1})}{\norm{P_{k-1}}^2},
\gamma_k = \frac{(x P_{k-1}, P_{k-2})}{\norm{P_{k-2}}^2}
$;
% TODO: Fragmenty dowodu.

% Wielomiany Legendre'a
\entry
Wielomiany Legendre'a:
ort. w $L^2(-1,1)$ z wagą $\rho = 1$:
$L_k(x) = \frac{2k-1}{k}xL_{k-1} - \frac{k-1}{k}L_{k-2}$,
$\norm{L_k}^2=\frac{2}{2k+1}$;

% Wielomiany Hermite'a.
\entry
Wielomiany Hermite'a:
$L^2_\rho(-\infty, \infty), \rho(x)=e^{-x^2}$,
$H_k(x) = 2xH_{k-1}(x) - (2k-2) H_{k-2}(x), H_0(x) \equiv 1$,
$\norm{H_k}^2 = \sqrt{\pi}2^kk!$;

% Wielomiany Czebyszewa.
\entry
Wielomiany Czebyszewa:
$L_\rho^2(-1,1), \rho(x)=\frac{1}{\sqrt{1-x^2}}$,
$T_k(x) = 2xT_{k-1}(x) - T_{k-2}(x), T_0(x)\equiv 1$,
$\norm{T_k}^2 = \pi \text{ if } k=0 \text{ else } \pi/2$;
 % Wykład 21.12.
\section{Aproksymacja jednostajna}

% Aproksymacja jednostajna, norma supremum.
\entry
$\norm{g}_\infty \coloneqq max_{x \in [a;b]} \abs{g(x)}$;

% Formuła trójczłonowa na wielomiany Czebyszewa.
% XXX: Czy jest poprawna?
\entry
Formuła trójczłonowa:
$T_{k + 1} = 2T_1(x)T_k(x) - T_{k-1}(x)$,
$T_1(x) = \cos(1 \cdot \theta)=x$,
$T_0(x) = 1$;

% Wniosek z formuły trójczłonowej.
\entry
$\abs{T_k(x)} \leq 1$
dla
$\abs{x} \leq 1$;

% Fakt o miejscach zerowych k-tego wielomianu Czebyszewa.
\entry
Miejsca zerowe $k$-tego wielomianu Czebyszewa są rzeczywiste, jednokrotne i przedziale $(-1;1)$: 
$x_j \coloneqq \cos(\frac{2j-1}{2k}\pi)$, 
$j=1,\ldots,k$;

% Fakt o lokalnych ekstremach k-tego wielomianu Czebyszewa.
\entry
Lokalne ekstrema $T_k$ w przedziale $[-1;1]$ są dane wzorem:
$y_j \coloneqq \cos \frac{j\pi}{k}$,
$j=1,\ldots,k$.
Pondato
$T_k(y_j) = (-1)^j$;

% Twierdzenie o własności minimaksu.
\entry
Tw. (własność minimaksu):
Spośród wszystkich wielomianów stopnia $k$ postaci
$1 \cdot x^k + o(x^k)$.
Wielomian 
$\frac{1}{2^{k-1}} T_k(x)$
ma najmniejszą normę maksimum na [-1;1].
Tzn.
$\max_{x \in [-1;1]} \abs{\frac{1}{2^{k-1}} T_k(x)} \leq 
\max_{x \in [-1;1]} \abs{w(x)} 
\forall w(x) = x^k + o(x^k)$;

% Węzły Czebyszewa w interpolacji wielomianowej Lagrange'a.
\entry
$\norm{f-w}_{\infty, [a,b]} \leq 
\frac{\norm{f^{(n+1)}}_{\infty, [a,b]}}{(n+1)!} 
\norm{(x-x_0) \cdots (x - x_n)}_\infty$, 
gdzie
$\max_{x\in[a,b] (x-x_0)\cdots(x-x_n)}$
jest najmniejszy dla 
$[a,b] = [-1,1]$,
gdy $x_i$ to miejsca zerowe $T_{n + 1}(x)$.
Dla dowolnego $[a,b]$ zachodzi analogiczny fakt dla węzłów Czebyszewa 
przeskalowanych liniowo na $[a,b]$:
$x \in [-1,1] \rightarrow [a,b] \ni \xi(x) = Ax + B$,
$x_i \mapsto \xi(x_i)$;

% Zadanie aproksymacji jednostajnej wielomianami.
\entry
$f \in C[a,b]$.
$\exists p^{*} \in \mathbb{P}_n$ 
najlepszej aproksymacji dla $f$ w sensie normy supremum na $[a,b]$, że
$\norm{f - p^{*}}_\infty \leq \norm{f - p}_\infty \forall p\in \mathbb{P}_n$;

% Twierdzenie Czebyszewa o alternansie
\entry
Tw. (Czebyszewa o alternansie):
$p^{*} \in \mathbb{P}_n$
jest ENA dla $f$
$\leftrightarrow$
istnieją co najmniej $n+2$ punkty $x_i$ w $[a,b]$, że
$f(x_i) - p^{*}(x_i) = \varepsilon \cdot (-1)^i \cdot \norm{f - p^{*}}_\infty$,
$\varepsilon \in \set{-1,+1}$;
($\abs{f(x_i) - p^{*}(x_i)} = \norm{f - p^{*}}_\infty$);

% Twierdzenie o jednoznaczności optymalnego wielomianowej aproksymacji jednostajnej.
\entry
Istnieje dokładnie 1 ENA dla zadanej 
$f\in C[a,b]$ 
w sensie aproksymacji jednostajnej;

% TODO: Wyznaczanie ENA dla wielomianowej jednostajnej nieskończonym algorytmem iteracyjnym Rameza.

% Fakt o błędzie najlepszej aproksymacji.
% TODO: Formuła barycentryczna na w(x).
\entry
$f \in C[-1,1]$,
$w$ --- w. i. L. dla $f$ opartej na w. Cz.:
$E_n \leq
\norm{f - w}_\infty \leq
(2 + 2 \ln(n + 1) / \pi)\cdot E_n <_{n \leq 10^5}
10\cdot E_n$;
 % Wykład 11.01.
\mnsection{Równania nieliniowe}

% TODO: Twierdzenie Darboux.

% TODO: Twierdzenie o przedziałach lokalizujących zero w kolejnych krokach metody bisekcji.

\entry
Szybkość zbieżności:
wykładnicza$(p)$:
$\exists_{c \geq 0} \exists_{N > 0}
\abs{x_{n + 1} - x^{*}} = c \abs{x_n - x^{*}}^p \forall_{n \geq N}$;
liniowa$(\gamma)$:
$\exists_{\gamma \in [0;1)} \exists_{N > 0}
\abs{x_{n + 1} - x^{*}} = \gamma \abs{x_n - x^{*}} \forall_{n \geq N}$;

% TODO: Metoda Newtona: czy tyle wystarczy?
\entry
Metoda Newtona:
$f(x) = 0 = f(x+h) = f(x) + f'(x)h + (f''(x)h^2/2+\cdots$;
$x_{n+1} = x_n - f(x_n)/f'(x_n)$;

% Twierdzenie o zbieżności metody Newtona.
\entry
Tw. (zbieżność m. N.):
$f \in C^2[a;b], \exists_{\ena{x} \in (a;b)}$,
że $f(\ena{x})=0, f'(\ena{x}) \neq 0$.
Wtedy $\exists_{U \ni \ena{x}}$, że m. N. zbieżna do $\ena{x} \ \forall_{x_0 \in U}$.
Ponadto zbieżność kwadratowa:
$\exists_{c>0} \abs{x_{n+1} - \ena{x}} \leq c \abs{x_n - \ena{x}}^2 \forall_{n=0,\ldots}$;

% Własności metody Newtona.
\entry
Własności m. N.:
\subentry
$f$ musi być różniczkowalna;
\subentry
musi być znany wzór na pochodną;
\subentry
$f'(\ena{x}) \neq 0 \Rightarrow \ena{x}$ zerem jednokrotnym i gdy krotności $m$, 
to metoda zbiega liniowo do 
$\abs{x_{n+1} - \ena{x}} \approx (1-1/m) \cdot \abs{x_n - \ena{x}}$;

% Metoda Newtona dla 1/a.
\entry
M. N. dla
$\frac{1}{a}$: $x_{n+1} = x_n(2 - x_na)$, $0<x_0<\frac{2}{a}$;

% TODO: Metoda Herona.

% TODO: Metoda siecznych.

% Metoda Steffensena.
\entry
M. Steffensena:
$x_{n+1} = x_n - f(x_n) \cdot \frac{f(x_n)}{f(x_n + f(x_n)) - f(x_n)}, p=2$,
(problemy z niedomiarem);

% TODO: Metoda cięciw.

% TODO: Metoda Halley'a.

% TODO: Metoda typu punktu stałego.

% TODO: Twierdzenie Banacha o kontrakcji
\entry
Tw. Banacha o kontrakcji:
$F:[a;b] \Rightarrow [a;b]$
i spełnia warunki kontrakcji:
$\exists_{\gamma < 1} \forallin{x,y}{[a;b]} F(\ena{x}) = \ena{x}$
oraz ciąg 
$x_{n+1} = F(x_n)$ 
jest zbieżny liniowo 
$\forallin{x_o}{[a;b]}$
do $\ena{x}$:
$\abs{x_{n+1} - \ena{x}} \leq \gamma \abs{x_n - \ena{x}}$;
 % Wykład 18.01.

\section{Sztuczki}

\entry
$Q^TQ=I$, to $\norm{Qx}^2_2=\norm{x}^2_2$;


\end{document}
