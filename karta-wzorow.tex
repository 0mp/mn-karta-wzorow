\documentclass[10pt,a4paper,twocolumn]{article}

\usepackage{multicol}
\usepackage{amsbsy, amssymb, latexsym, amsmath, braket}
\usepackage[tiny]{titlesec}
\usepackage[hmargin=0.5cm,vmargin=1.0cm]{geometry}
\usepackage[utf8x]{inputenc}
\usepackage{polski}
\usepackage{scalefnt}
\usepackage[yyyymmdd,hhmmss]{datetime}
\usepackage{commath}
\usepackage{mathtools}
\usepackage{tikz}
\usetikzlibrary{tikzmark}

% Definicje.
% https://tex.stackexchange.com/a/138901/80219
\newcommand{\Hsquare}{%
  \text{\fboxsep=-.2pt\fbox{\rule{0pt}{1ex}\rule{1ex}{0pt}}}%
}

% Hack, który sprawia, że \cdot w wyrażeniu \norm{A}\cdot\norm{A} znajduje się
% w równej odległości od obu norm. W przeciwnym wypadku \cdot znajduje się
% dziwnie blisko prawej normy.
% https://tex.stackexchange.com/a/99882/80219
% https://tex.stackexchange.com/a/61511/80219
\DeclareRobustCommand*{\norm}[1]{\|#1\|}

\newcommand{\entry}{$\bullet$\hspace{0.15em}}
\newcommand{\subentry}{$\circledcirc$\hspace{0.15em}}
% https://tex.stackexchange.com/a/7045/80219
\newcommand{\textsubentry}[1]{\tikz[baseline=(char.base)]{
            \node[shape=circle,draw,inner sep=1pt] (char) {#1};\hspace{0.15em}}}

\titlespacing{\section}{0pt}{0pt}{0pt}
\titlespacing{\subsection}{0pt}{0pt}{0pt}
\titlespacing{\subsubsection}{0pt}{0pt}{0pt}

% Wyłącz numerowanie stron.
\pagenumbering{gobble}

\setlength{\parindent}{0pt}
% Odległość pomiędzy liniami. Zmniejsz, jeżeli brakuje miejsca.
\setlength{\parskip}{0.5ex}

\title{Karta wzorów z metod numerycznych}

\begin{document}
% Rozmiar czcionki.
\scalefont{.8}

\text{\tiny{
    Wersja
    \input{|"git rev-list --count --first-parent HEAD"}z
    \today\ o \currenttime\ (\pdfmdfivesum file{./karta-wzorow.tex})
}}

\mnsection{Układy równań liniowych}

\entry
Metody podstawienia w przód/tył. Koszt: $O(2n^2)$;

% Rozwiązywanie układu z macierzą ortogonalną.
\entry
$Q^TQ=I$, $Qx=b$, to $Q^TQx=Q^Tb$, to $x=Q^Tb$ w $O(n^2)$;

% TODO: Układ blokowy [A1 E; F C] * [x; y] = [f; g].

% TODO: Układ, w którym znany jest rozkład macierzy.
% A=BC, gdzie B, C --- macierze łatwe. Wówczas By=b, Cx=y

% TODO: Algorytm rozkładu LU (eliminacji Gaussa z wyborem elementu głównego w kolumnie).
\entry
Metoda el. Gaussa (LU), gdy $A$ nieosobliwa. Koszt: $O(\frac{2}{3}n^3)$;

\entry
$A$ dodatnio określona $\iff \forall_{x \neq 0} x^TAx>0$;

% TODO: Zapis algorytmu.
\entry
R. Chol.:
$A=A^T>0$,
to
$A=LL^T$;
$O(\frac{1}{3}n^3)$;
$A=LDL^T$
nie używa
$\sqrt{\cdot}$;

% Losowe fakty na temat macierzy.
\entry
$AA^T$ jest symetryczna;
\entry
$A$ odwracalna, to $AA^T>0$;

\section{Arytmetyka fl}

% TODO: Dokładniejszy opis.
\entry
Liczby maszynowe: $x=(-1)^s \cdot m \cdot \beta^e$, gdzie $0\leq m < \beta$ i $m=(f_0, \ldots, f_{p-1})_\beta$, $f_i\in\set{0,1,\ldots,\beta-1}$, gdzie $p$ to precyzja arytmetyki;

\entry
Liczby znormalizowane: $x=(-1)^s\cdot m \cdot 2^e$, gdzie $1\leq m < 2$;

\entry
Gdy $x$ l. maszynową, to $\mathrm{fl}(x)=x$, wpp. $\mathrm{fl}(x)=\mathrm{ROUND}(x)=\mathrm{RN}(x)$, gdzie $\mathrm{RN}(x)$ zaokrągla do najbliższej l. maszynowej;

\entry
Błąd reprezentacji: $\abs{\mathrm{fl}(x) - x}$;

\entry
$\frac{\abs{\mathrm{fl}(x) - x}}{\abs{x}}\leq 2^{-p}=\nu$, o ile nie ma *flow;

\entry
$\mathrm{fl}(x) = x(1+\epsilon_x)$, $\abs{\epsilon_x}\leq \nu$

\entry
Standard fl gwarantuje, że $a \Hsquare b = \mathrm{fl}(a\Hsquare b)$, jeżeli nie ma NaN, *flow;

\entry
Fused multiply-add:
$\mathrm{fl}(\mathrm{FMA}(a,b,c)) = \mathrm{fl}(a \cdot b + c)$;

% Błędy w obliczeniach numerycznych.

\entry
Błąd bezwzględny $=\norm{\tilde{x} - x}$;

\entry
Błąd względny $=\frac{\text{błąd bezwzgl.}}{\norm{x}}$;

\entry
Silna numeryczna poprawność (NP): algorytm $A$ jest silnie numerycznie poprawny (NP), jeśli $\tilde{y} = P(\tilde{x})$, gdzie $\norm{\tilde{x}-x} / x\leq k\cdot\nu$, $P$---zadanie, $\tilde{y}$---dokładne rozw. dla trochę zaburzonych danych;

\entry
Słaba numeryczna poprawność (stabilność): daje prawie dokł. rozw. dla prawie dokł. danych: $\norm{\tilde{x}-x} / x\leq k\cdot\nu$, $\norm{\tilde{y}-P(\tilde{x})} / P(\tilde{x})\leq k\cdot\nu$;

\section{Normy}

% Normy wektorowe w $\mathbb{R}^n$.

\entry
$\norm{x}_1 \coloneqq \sum^n_{i=1}\abs{x_i}$;
\entry
$\norm{x}_2 \coloneqq \sqrt{x_1^2+\ldots + x_n^2}$;

\entry
$\norm{x}_\infty \coloneqq \max_{i}\abs{x_i}$;
\entry
$\norm{x}_p \coloneqq (\sum_i\abs{x_i}^p)^{1/p}$;

% Nierówność Cauchy’ego-Schwarza.
\entry
$\abs{x^Ty}_2 \leq \norm{x}_2\cdot\norm{y}_2$, dla $x,y\in\mathbb{R^N}$;

\entry
$Q^TQ=I$, to $\norm{Qx}^2_2=\norm{x}^2_2$;

% Normy macierzowe indukowane przez normy wektorowe.

\entry
$\norm{A}_p \coloneqq \max\limits_{x\neq 0} \frac{\norm{A_x}_p}{\norm{x}_p} = \max\limits_{\norm{x}_p = 1} \norm{A_x}_p$;

% Fakty.
\entry
$\norm{I} = 1$;
\entry
$\norm{Ax} \leq \norm{A} \cdot \abs{x}$;
\entry
$\norm{AB} \leq \norm{A} \cdot \norm{B}$;

\entry
$\norm{A}_1 \coloneqq \max_j\sum_i\abs{a_{ij}}$;
\entry
$\norm{A}_\infty \coloneqq \max_i\sum_j\abs{a_{ij}}$;

\entry
$\norm{A}_2 \coloneqq \max\set{\sqrt{\lambda} : \lambda\text{ --- wartość własna } A^TA}$;

\entry
$\norm{D} = \max_i d_i$;

\entry
Jeśli $\forall_{x} \norm{Ax}\leq c\norm{x}$, że $\exists_y \norm{Ay} = c\norm{y}$, to $c=\norm{A}$;

% Fakt z zadania 2.4. z części 2 ćwiczeń Konrada Sakowskiego.
\entry
$\norm{A}_2 = \sup_{\norm{x}_2=1, \norm{y}_2=1}\abs{y^TAx}$;

% Fakt z zadania 2.5. z części 2 ćwiczeń Konrada Sakowskiego.
\entry
$A\in\mathbb{R}^{N\times N} \Rightarrow \norm{A}^2_2=\norm{A^TA}_2$;

% Normy macierzowe nieindukowane przez normy wektorowe.

\entry
Norma Frobeniusa: $\norm{A}_F \coloneqq \sqrt{\sum_i\sum_j\abs{a_{ij}}^2}$; $\norm{I}_F=\sqrt{N}$;



\section{Uwarunkowanie zadania}

\entry
Wskaźnik uwarunkowania $P$ w punkcie $x$:\\ $\mathrm{cond}_{\mathrm{abs}}(P,x) \coloneqq \sup\limits_{\text{małe }\delta} \frac{\norm{P(x+\delta) - P(x)}}{\norm{\delta}}$;

\entry
$\norm{P(x+\delta)} \leq \mathrm{cond}_{\mathrm{abs}}(P,x)\norm{\delta}$;

\entry
$\mathrm{cond}_{\mathrm{rel}}(P,x) \coloneqq \sup\limits_{\text{małe }\delta} \frac{\norm{P(x+\delta) - P(x)} \cdot \norm{x}}{\norm{\delta} \cdot \norm{P(x)}}$;

\entry
$\frac{\norm{P(x+\delta) - P(x)}}{\norm{P(x)}} \leq \mathrm{cond}_{\mathrm{rel}}(P,x)\frac{\norm{\delta}}{\norm{x}}$;

% Idealizacja.

\entry
$\mathrm{cond}_{\mathrm{abs}}(P,x) \coloneqq \lim\limits_{\norm{\delta \to 0}} \frac{\norm{P(x+\delta) - P(x)}}{\norm{\delta}}$;

\entry
$\mathrm{cond}_{\mathrm{rel}}(P,x) \coloneqq \mathrm{cond}_{\mathrm{abs}}(P,x)\frac{\norm{x}}{\norm{P(x)}} $;

\entry
Zadanie $P$ jest źle uwarunkowane w punkcie $x$, gdy $\mathrm{cond}(P,x) \gg 1$, bo małe zaburzenie danych może spowodować duży błąd wyniku;

\entry
Uwarunkowanie macierzy: $\mathrm{cond}(A) \coloneqq \norm{\vphantom{A^{-1}}A}\cdot\norm{A^{-1}}$;

\entry
Gdy $\epsilon\cdot\norm{\vphantom{A^1}A}\cdot\norm{A^{-1}} < \frac{1}{2}$, to $\frac{\norm{x - \tilde{x}}}{\norm{x}} \leq 4 \cdot \mathrm{cond}(A) \cdot \epsilon$;

\entry
Jeśli $\norm{B} < 1$, to $I+B$ odwracalna i $\norm{(I+B)^{-1}} \leq 1 / (1 - \norm{B})$;

\entry
Jeśli $A$ odwracalna i $\norm{A^{-1}\Delta} < 1$,\\ to $(A+\Delta)^{-1}$ istnieje i $\norm{(A+\Delta)^{-1}} \leq \frac{\norm{A^{-1}}}{1 - \norm{A^{-1}} \cdot \norm{\vphantom{A^1}\Delta}}$, gdzie $\Delta \leq \epsilon\norm{A}$;

% TODO: Fakt o rozwiązaniu \tilde{x} układu Ax=b za pomocą LU z wyborem w kolumnie.

% TODO: Numeryczne kryterium NP.

% TODO: Przykłady zadań dobrze i źle uwarunkowanych.

% \section{Macierze rzadkie}

\entry
Metody bezpośrednie: r. LU z permutacją wierszy i kolumn, aby w czynnikach LU było dużo zer;

% TODO: Przykład strzałki Wilkinsona.

% Metody iteracyjne / stacjonarne.
\subsection{Metody stacjonarne}

\entry
$A=M-Z$, $Ax=Mx-Zx=b$, $Mx=b+Zx$, $Mx_{k+1}=b+Zx_{k}$ (iteracja prosta);

\entry
Metoda (**): $x_{k+1}=M^{-1}(b+Zx_k)$;

\entry
Chcemy, aby $Mx_{k+1}=\tilde{b}$ było łatwe;

\entry
Tw.: $\norm{M^{-1}Z} < 1$, to $\forall x_0 \in \mathbb{R}^N$ metoda (**) zbieżna do rozw. $Ax=b$;

\entry
Tw.: $\forall x_0$ metoda (**) zbieżna do $x^* \leftrightarrow \rho(M^{-1}Z) < 1$, gdzie $\rho(B) \coloneqq \max\set{\abs{\lambda} : \lambda\text{---w. wł. } B}$;

% XXX: Da się więcej powiedzieć.
\entry
Metoda Jacobiego: $M=\mathrm{diag}(A) = D$; zbieżna, gdy $A$ ma dominująca przekątną, tzn. $\abs{a_{ii}} > \sum_{j\neq i}\abs{a_{ij}} \forall i=1\ldots N$;

% TODO: Metoda Gaussa-Seidla.

% TODO: Metody przestrzeni Kryłowa.

% TODO: Metoda gradientów sprzężonych (CG).

% Metody iteracyjne / stacjonarne.
\section{Metody stacjonarne}

% Ogólna idea metod stacjonarnych (1).
\entry
$A=M-Z$;
\entry
$Mx_{k+1}=b+Zx_{k}$
\entry
$x_{k+1}=x_k+M^{-1}r_k$, gdzie $r_k=b-Ax_k$;

% Ogólna idea metod stacjonarnych (2).
\entry
Metoda: $\mathrm{(**)}$ $x_{k+1}=M^{-1}(b+Zx_k)$,
aby $Mx_{k+1}=\tilde{b}$ było łatwe;

% Twierdzenie o zbieżności metody iteracyjnej (1).
\entry
Tw.: $\norm{M^{-1}Z} < 1$,
to metoda $\mathrm{(**)}$ zbieżna do rozw. $Ax=b \forall x_0 \in \mathbb{R}^N$

% Twierdzenie o zbieżności metody iteracyjnej (2).
\entry
Tw.: $\forall x_0$,
metoda $\mathrm{(**)}$
zbieżna do $x^* \leftrightarrow \rho(M^{-1}Z) < 1$,
gdzie $\rho(B) \coloneqq \max\set{\abs{\lambda} : \lambda\text{ --- w. wł. } B}$;

% Przykładowy algorytm implementujący iterację.
\entry
$\mathrm{iter}(A,b,x_0,M,t)$:
$x=x_0;
r=b - Ax;
(L,U,P) = lu(M);
\mathrm{while}(\norm{r} > t) \{
    p = M^{-1}r;
    x=x+p;
    r=b-Ax
\};
\mathrm{return}(x)
$;
$O(N^3 + \mathrm{\#iter} \cdot N^2)$;

% Metoda Jacobiego.
\entry
Metoda Jacobiego:
$M=\mathrm{diag}(A) = D$, $A=L+D+U$;
zbieżna, gdy $A$ ma dominująca przekątną,
tzn. $\abs{a_{ii}} > \sum_{j\neq i}\abs{a_{ij}} \forall i=1, \ldots, N$;

% Metoda Gaussa-Seidla.
\entry
Metoda Gaussa-Seidla:
$M = L + D$;
zbieżna, gdy $a_{ii}x_i^{k+1}=b_i-\sum_{j>i}a_{ij}x^k-\sum_{j<i}a_{ij}x^{k+1}$;

% Metoda Richardsona.
\entry
Metoda Richardsona:
$x_k+1 = x_k + \tau(b-Ax_k)$, gdzie $\tau\in\mathbb{R}$.
Gdy $A=A^T>0$,
to $\tau_{\mathrm{opt}}=\frac{2}{\lambda_{\mathrm{min}} + \lambda_{\mathrm{max}} }$
i $\gamma \coloneqq \norm{M^{-1}Z}=\abs{1-\lambda_{\mathrm{max}}\tau_{\mathrm{opt}}}=\frac{\mathrm{cond}_2(A)-1}{\mathrm{cond}_2(A) + 1}$;

% Algorytm iteracyjnego poprawiania w modelowej sytuacji.
\entry
Modelowe IR:
$\mathrm{for}(n=0,\ldots)\{ r_n=b-Ax_n; Ae_n=_{\mathrm{fl}}r_n; x_{n+1} = x_n + e_n \}$;

% TODO: Dodać dowody zbieżności modelowego algorytmu.

% Zbieżność modelowego algorytmu iteracyjnego poprawiania rozwiązania (IR, czyli iterative reconstruction)
\entry
Jeżeli $\mathrm{cond}(A)$ jest niezbyt duże względem $\frac{1}{\nu}$, to modelowy algorytm IR jest zbieżny;


\section{Metody przestrzeni Kryłowa}

% Definicja iteracji w metodach przestrzeni Kryłowa.
\entry
$k$-ta iteracja:
$x_k\in x_o+K_k$,
gdzie $K_k\coloneqq \set{r_0, Ar_0,\ldots, A^{k-1}r_0}$,
gdzie $r_0=b-Ax_0$;

% TODO: Coś tu jest nie tak.
% Metody oparte na minimalizacji błędu.
\entry
Metody oparte na minimalizacji błędu:
$x_k\in x_0 +K_k$ oraz $\norm{x_k - x^*}_B \forall x\in x_0 + K_k$;

% Metody oparte na minimalizacji residuum.
\entry
Metody oparte na minimalizacji residuum:
$\norm{b-Ax_k}_B \leq \norm{b-Ax}_B \forall x\in x_0 +K_k$, gdzie $B=B^T>0$;

% Metoda gradientów sprzężonych (CG).
\entry
Metoda gradientów sprzężonych (CG):
$A=A^T>0$, $x_k\in x_0 + K_k$ t.,
że $\norm{x_k+x^*}_A\leq \norm{x-x^*}_A \forall x\in x_0 + K_k$,
gdzie $\norm{y}^2_A \coloneqq y^TAy$.
Koszt iteracji, to mnożenie wektora przez $A$, czyli $O(N)$.
W idealnej arytmetyce zbieżne do $x^*$ w $\leq N$ iteracjach.
Po $k$ iteracjach:
$\norm{x_k - x^*}_A\leq 2(\frac{\sqrt{H} - 1}{\sqrt{H} + 1})^k\norm{x_0-x^*}_A$,
gdzie $H=\mathrm{cond}_2(A)$;

% TODO: Wzmianka o GMRES.

\section{LZNK}

% TODO: Opisać algorytm iteracyjnego poprawiania w modelowej sytuacji.

\entry
Jeżeli $\mathrm{cond}(A)$ jest niezbyt duże względem $\frac{1}{\nu}$, to modelowy algorytm IR jest zbieżny;

\entry
LZNK: układ $n$ równań z $m$ niewiadomymi.
Cel: $\norm{b - Ax}_2 \to \min !$;
% $\sum_i(\sum_ja_{ij}-b_i)^2 \to \min !$

% TODO: Dodać rysunki macierzy.
\entry
Rozkład QR:
$A = QR = Q_1R_1$;

\entry
Algorytm rozwiązywania LZNK przez rozkład QR:
\textsubentry{1}
$A=QR$
\textsubentry{2}
$\norm{b-Ax}^2_2 = \norm{Q^T_1b-R_1x}^2_2 + \norm{\tilde{Q}^Tb}^2_2 \to \min !$
\textsubentry{3}
Rozwiązaniem jest $x\in \mathbb{R}^m$ spełniający ukł. r-ń z macierzą trójkątną $R_1x=Q_1^Tb$.
Koszt: $O(m^2)$.
Koszt $Q_1^Tb$: $O(nm)$;

% TODO: Wyznaczanie rozkładu QR metodą Householdera.
% TODO: Definicja przekształcenia Householdera.
% TODO: Odbicie Householdera.

\entry
Koszt rozkładu QR met. Householdera: $T(n,m)=2m^2n - \frac{2}{3}m^3$;

\section{Zadanie własne}

% TODO: Metody numeryczne rozwiązywania zagadnienia własnego.

% TODO: Metoda potęgowa.

\entry
Zadanie własne $A \in \mathbb{R}^{N \times N}$:
$(\lambda,x) \in \mathbb{C} \times \mathbb{C}^N: Ax = \lambda x$;

\entry
Gdy $\lambda$ dla $A$, to $(\lambda - \mu)$ dla $A-\mu I$;

\entry
Gdy $\lambda$ dla $A$ nieos., to $\frac{1}{\lambda}$ dla $A^{-1}$;

% TODO: Wniosek z powyższych faktów dla metody potęgowej.

% ^^^^^^^^^^^^^^^^^^^^^^^^^^^^^
% Zakres materiału do kolokwium.

% TODO: Odwrotna metoda potęgowa.

% TODO: Metoda Rayleigh (RQI).

% TODO: Fakty o lokalizacji wartości własnych.

% TODO: Twierdzenie Gershgorina.

% TODO: Twierdzenie (Bauer-Fike).

% TODO: Metoda QR wyznaczania wszystkich wartości własnych macierzy (wraz z ulepszeniem i wariantem praktycznym).

% \mnsection{Sztuczki}

% Całkowanie przez części
\entry
$\int f(x)g'(x)dx = f(x)g(x) - \int f'(x)g(x)dx$;


\section{Aproksymacja jednostajna}

% Aproksymacja jednostajna, norma supremum.
\entry
$\norm{g}_\infty \coloneqq max_{x \in [a;b]} \abs{g(x)}$;

% Formuła trójczłonowa na wielomiany Czebyszewa.
% XXX: Czy jest poprawna?
\entry
Formuła trójczłonowa:
$T_{k + 1} = 2T_1(x)T_k(x) - T_{k-1}(x)$,
$T_1(x) = \cos(1 \cdot \theta)=x$,
$T_0(x) = 1$;

% Wniosek z formuły trójczłonowej.
\entry
$\abs{T_k(x)} \leq 1$
dla
$\abs{x} \leq 1$;

% Fakt o miejscach zerowych k-tego wielomianu Czebyszewa.
\entry
Miejsca zerowe $k$-tego wielomianu Czebyszewa są rzeczywiste, jednokrotne i przedziale $(-1;1)$: 
$x_j \coloneqq \cos(\frac{2j-1}{2k}\pi)$, 
$j=1,\ldots,k$;

% Fakt o lokalnych ekstremach k-tego wielomianu Czebyszewa.
\entry
Lokalne ekstrema $T_k$ w przedziale $[-1;1]$ są dane wzorem:
$y_j \coloneqq \cos \frac{j\pi}{k}$,
$j=1,\ldots,k$.
Pondato
$T_k(y_j) = (-1)^j$;

% Twierdzenie o własności minimaksu.
\entry
Tw. (własność minimaksu):
Spośród wszystkich wielomianów stopnia $k$ postaci
$1 \cdot x^k + o(x^k)$.
Wielomian 
$\frac{1}{2^{k-1}} T_k(x)$
ma najmniejszą normę maksimum na [-1;1].
Tzn.
$\max_{x \in [-1;1]} \abs{\frac{1}{2^{k-1}} T_k(x)} \leq 
\max_{x \in [-1;1]} \abs{w(x)} 
\forall w(x) = x^k + o(x^k)$;

% Węzły Czebyszewa w interpolacji wielomianowej Lagrange'a.
\entry
$\norm{f-w}_{\infty, [a,b]} \leq 
\frac{\norm{f^{(n+1)}}_{\infty, [a,b]}}{(n+1)!} 
\norm{(x-x_0) \cdots (x - x_n)}_\infty$, 
gdzie
$\max_{x\in[a,b] (x-x_0)\cdots(x-x_n)}$
jest najmniejszy dla 
$[a,b] = [-1,1]$,
gdy $x_i$ to miejsca zerowe $T_{n + 1}(x)$.
Dla dowolnego $[a,b]$ zachodzi analogiczny fakt dla węzłów Czebyszewa 
przeskalowanych liniowo na $[a,b]$:
$x \in [-1,1] \rightarrow [a,b] \ni \xi(x) = Ax + B$,
$x_i \mapsto \xi(x_i)$;

% Zadanie aproksymacji jednostajnej wielomianami.
\entry
$f \in C[a,b]$.
$\exists p^{*} \in \mathbb{P}_n$ 
najlepszej aproksymacji dla $f$ w sensie normy supremum na $[a,b]$, że
$\norm{f - p^{*}}_\infty \leq \norm{f - p}_\infty \forall p\in \mathbb{P}_n$;

% Twierdzenie Czebyszewa o alternansie
\entry
Tw. (Czebyszewa o alternansie):
$p^{*} \in \mathbb{P}_n$
jest ENA dla $f$
$\leftrightarrow$
istnieją co najmniej $n+2$ punkty $x_i$ w $[a,b]$, że
$f(x_i) - p^{*}(x_i) = \varepsilon \cdot (-1)^i \cdot \norm{f - p^{*}}_\infty$,
$\varepsilon \in \set{-1,+1}$;
($\abs{f(x_i) - p^{*}(x_i)} = \norm{f - p^{*}}_\infty$);

% Twierdzenie o jednoznaczności optymalnego wielomianowej aproksymacji jednostajnej.
\entry
Istnieje dokładnie 1 ENA dla zadanej 
$f\in C[a,b]$ 
w sensie aproksymacji jednostajnej;

% TODO: Wyznaczanie ENA dla wielomianowej jednostajnej nieskończonym algorytmem iteracyjnym Rameza.

% Fakt o błędzie najlepszej aproksymacji.
% TODO: Formuła barycentryczna na w(x).
\entry
$f \in C[-1,1]$,
$w$ --- w. i. L. dla $f$ opartej na w. Cz.:
$E_n \leq
\norm{f - w}_\infty \leq
(2 + 2 \ln(n + 1) / \pi)\cdot E_n <_{n \leq 10^5}
10\cdot E_n$;
 % Wykład 11.01.
\mnsection{Równania nieliniowe}

% TODO: Twierdzenie Darboux.

% TODO: Twierdzenie o przedziałach lokalizujących zero w kolejnych krokach metody bisekcji.

\entry
Szybkość zbieżności:
wykładnicza$(p)$:
$\exists_{c \geq 0} \exists_{N > 0}
\abs{x_{n + 1} - x^{*}} = c \abs{x_n - x^{*}}^p \forall_{n \geq N}$;
liniowa$(\gamma)$:
$\exists_{\gamma \in [0;1)} \exists_{N > 0}
\abs{x_{n + 1} - x^{*}} = \gamma \abs{x_n - x^{*}} \forall_{n \geq N}$;

% TODO: Metoda Newtona: czy tyle wystarczy?
\entry
Metoda Newtona:
$f(x) = 0 = f(x+h) = f(x) + f'(x)h + (f''(x)h^2/2+\cdots$;
$x_{n+1} = x_n - f(x_n)/f'(x_n)$;

% Twierdzenie o zbieżności metody Newtona.
\entry
Tw. (zbieżność m. N.):
$f \in C^2[a;b], \exists_{\ena{x} \in (a;b)}$,
że $f(\ena{x})=0, f'(\ena{x}) \neq 0$.
Wtedy $\exists_{U \ni \ena{x}}$, że m. N. zbieżna do $\ena{x} \ \forall_{x_0 \in U}$.
Ponadto zbieżność kwadratowa:
$\exists_{c>0} \abs{x_{n+1} - \ena{x}} \leq c \abs{x_n - \ena{x}}^2 \forall_{n=0,\ldots}$;

% Własności metody Newtona.
\entry
Własności m. N.:
\subentry
$f$ musi być różniczkowalna;
\subentry
musi być znany wzór na pochodną;
\subentry
$f'(\ena{x}) \neq 0 \Rightarrow \ena{x}$ zerem jednokrotnym i gdy krotności $m$, 
to metoda zbiega liniowo do 
$\abs{x_{n+1} - \ena{x}} \approx (1-1/m) \cdot \abs{x_n - \ena{x}}$;

% Metoda Newtona dla 1/a.
\entry
M. N. dla
$\frac{1}{a}$: $x_{n+1} = x_n(2 - x_na)$, $0<x_0<\frac{2}{a}$;

% TODO: Metoda Herona.

% TODO: Metoda siecznych.

% Metoda Steffensena.
\entry
M. Steffensena:
$x_{n+1} = x_n - f(x_n) \cdot \frac{f(x_n)}{f(x_n + f(x_n)) - f(x_n)}, p=2$,
(problemy z niedomiarem);

% TODO: Metoda cięciw.

% TODO: Metoda Halley'a.

% TODO: Metoda typu punktu stałego.

% TODO: Twierdzenie Banacha o kontrakcji
\entry
Tw. Banacha o kontrakcji:
$F:[a;b] \Rightarrow [a;b]$
i spełnia warunki kontrakcji:
$\exists_{\gamma < 1} \forallin{x,y}{[a;b]} F(\ena{x}) = \ena{x}$
oraz ciąg 
$x_{n+1} = F(x_n)$ 
jest zbieżny liniowo 
$\forallin{x_o}{[a;b]}$
do $\ena{x}$:
$\abs{x_{n+1} - \ena{x}} \leq \gamma \abs{x_n - \ena{x}}$;
 % Wykład 18.01.

\end{document}
